%% Согласно ГОСТ Р 7.0.11-2011:
%% 5.3.3 В заключении диссертации излагают итоги выполненного исследования, рекомендации, перспективы дальнейшей разработки темы.
%% 9.2.3 В заключении автореферата диссертации излагают итоги данного исследования, рекомендации и перспективы дальнейшей разработки темы.
В рамках данной диссертационной работы методом численного моделирования были исследованы закономерности захвата и десорбции дейтерия в вольфраме при импульсных плазменном и лазерном воздействии. Среди наиболее значимых результатов можно выделить следующие:
\begin{enumerate}
  \item В программном пакете FESTIM была реализована модель, учитывающая кинетику процессов на поверхности. Реализованная модель расширяет функциональные возможности кода, что подтверждено в ходе проверки корректности ее имплементации. Модель включена в состав свободно распространяемого программного обеспечения и доступна всем пользователям.
  \item Сравнение результатов моделирования с экспериментальными данными по захвату дейтерия в вольфраме при импульсном плазменном облучении и экспериментами по ЛИД дейтерия из вольфрамовых пленок показало хорошее соответствие, что подтверждает применимость стандартных моделей для анализа динамики транспорта дейтерия при импульсных нагрузках.
  \item На основе численного моделирования проведен комплексный анализ влияния импульсно-периодических плазменных нагрузок, соответствующих ELM-событиям в крупных токамаках, на долговременное накопление дейтерия в вольфраме. Установлено, что в широком диапазоне параметров облучения скорость накопления дейтерия в переходных процессах снижается вследствие существенного нагрева материала высокоэнергетичными частицами. Эффект усиливается с ростом частоты импульсных нагрузок. 
  \item Показано, что дополнительный нагрев во время переходных процессов способствует более глубокому проникновению дейтерия. Данный эффект усиливается с увеличением частоты импульсных нагрузок, приводящих к росту средней температуры материала. В условиях термоядерных установок это может влиять как на скорость проникновения изотопов водорода в систему охлаждения, так и на эффективность дегазации элементов установки между плазменными кампаниями.
  \item В приближении малых импульсных нагрузок во время переходных событий построена аналитическая модель, описывающая квазистационарное распределение дейтерия в материале с учетом влияния центров захвата и градиента температур. Развитая модель может быть использована для оценки предельного уровня содержания изотопов водорода в квазистационарном режиме при достижении насыщения.
  \item Проведен детальный численный анализ состава потока десорбированных частиц при ЛИД. Установлено, что вероятность прямой десорбции атомов увеличивается с ростом температуры поверхности и уменьшением полного потока выходящих частиц. Условия для десорбции атомов могут быть выполнены при использовании лазерных импульсов с длительностью от \SI{10}{\micro\second}. В рамках подхода получена оценка атомарной фракции на уровне \( \sim \SI{10}{\percent} \) в случае использования лазерных импульсов с миллисекундной длительностью, что может вносить дополнительную погрешность при проведении оценки содержания изотопов водорода.
  \item Исследована эффективность анализа содержания дейтерия методом ЛИД в широком диапазоне параметров материала и лазерного облучения. Наибольшая эффективность достигается при использовании более длительных лазерных импульсов. Показано, что измерение температуры поверхности позволяет снизить влияние неопределенности теплофизических свойств материала на точность измерений. Основным источником погрешности является неопределенность энергетического барьера выхода из центров захвата, тогда как влияние концентрации центров захвата оказывается менее значительным.    
\end{enumerate}
