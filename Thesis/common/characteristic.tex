
{\actuality} Одним из перспективных направлений формирования новых источников энергии является управляемый термоядерный синтез (УТС). 
Наибольшие успехи на пути к практической реализации УТС достигнуты в установках с магнитным удержанием горячей плазмы типа токамак. 
В настоящее идет активная фаза строительства международного экспериментального реактора ИТЭР, спроектированного для демонстрации 
возможности квазистационарного (400 с) удержания термоядерной плазмы, а также введены в эксплуатацию наибольший в России токамак "--- Т15-МД, 
и наибольший в мире токамак JT60-SA, расположенный в Японии. Помимо этого, во множестве стран разрабатываются проекты установок следующего 
поколения для отработки реакторных технологий, в том числе в России ведется активное проектирование токамака ТРТ. Реализация термоядерной реакции 
планируется на основе дейтерий-тритиевой смеси. Использование радиоактивного трития накладывает определенные ограничения на эксплуатацию установки 
с точки зрения радиационной безопасности. В связи с этим, одной из важнейших задач будущих термоядерных установок является систематический контроль 
за накоплением трития в обращенных к плазме элементах (ОПЭ). 
 
 В качестве основного материала ОПЭ рассматривается вольфрам. Согласно оценкам~\cite{Roth1}, захват трития в вольфрамовые элементы не будет определяющим 
 процессом глобального удержания во время нормальных режимов работы установки. Однако в режимах с улучшенным удержанием горячей плазмы (H-мода) протекают 
 переходные процессы, например ELM-неустойчивости (Edge Localised Modes), приводящие к периодическим импульсным потокам высокой мощности на поверхность ОПЭ. 
 Результаты экспериментов по имитации воздействия мощных плазменных потоков в линейной плазменной установке КСПУ-Т~\cite{Ogorodnikova} показывают, что скорость 
 накопления изотопов водорода может оказаться выше, чем в случае стационарного облучения, характерного для нормальных плазменных разрядов. Однако параметры 
 облучения в установках такого типа не могут в полной мере воспроизвести условия, соответствующие крупным токамакам. Ввиду этого, исследование закономерностей 
 накопления и удержания изотопов водорода под действием мощных импульсно-периодических плазменных нагрузок представляет повышенный интерес.
 
Схожей задачей является дистанционная диагностика содержания изотопов водорода в ОПЭ при помощи лазерно-индуцированной десорбции (ЛИД). Метод ЛИД 
заключается в нагреве участка исследуемой поверхности лазерным импульсом с последующим анализом состава вышедшего газа. Данная диагностика апробируется 
на сферическом токамаке Глобус-М2. Помимо этого, возможность применения ЛИД является одной из приоритетных задач исследований ИТЭР~\cite{loarte2020required}, 
а также рассматривается для российского проекта ТРТ~\cite{Razdobarin2022}.

{\aim} данной работы является выявление закономерностей удержания изотопов водорода в вольфраме под действием импульсных плазменных и тепловых нагрузок.

Для~достижения поставленной цели необходимо было решить следующие {\tasks}:
\begin{enumerate}[beginpenalty=10000] % https://tex.stackexchange.com/a/476052/104425
    \item Исследование динамики удержания изотопов водорода в вольфраме при облучении мощными импульсными потоками плазмы, соответствующими быстрым переходным процессам, типа ELM-неустойчивости в токамаках;
    \item Исследование влияния быстрых переходных процессов, типа ELM-неустойчивости в токамаках, на рециклинг изотопов водорода;
    \item Оценка эффективности различных режимов ЛИД для анализа содержания изотопов водорода в ОПЭ;
    \item Валидация модели ЛИД.
\end{enumerate}


{\novelty}
\begin{enumerate}[beginpenalty=10000] % https://tex.stackexchange.com/a/476052/104425
  \item Впервые \ldots
  \item Впервые \ldots
  \item Было выполнено оригинальное исследование \ldots
\end{enumerate}

{\influence} \ldots

{\methods} Достижение поставленной цели и решение сопутствующих задач осуществляется путем проведения численного моделирования, 
позволяющего оценить влияние импульсных нагрузок в широком диапазоне параметров, обычно недоступном в рамках действующих экспериментальных 
и лабораторных установок. В качестве основного численного метода для анализа динамики удержания и транспорта изотопов водорода в ОПЭ применяется 
метод конечных элементов, реализованный в программных пакетах COMSOL Multiphysics и FESTIM. Участие в разработке 
последнего также проходит в рамках данной работы.

{\defpositions}
\begin{enumerate}[beginpenalty=10000] % https://tex.stackexchange.com/a/476052/104425
  \item Аналитическая модель, описывающая стационарное распределения водорода в материале при учете эффекта Соре (градиента температур) 
  и ловушек водорода в приближении мгновенной рекомбинации водорода на обращенной к плазме поверхности, позволяющая прогнозировать предельное 
  накопление изотопов водорода в обращенных к плазме материалах;
  \item Возникновение мощных, импульсно-периодических, плазменных нагрузок (частота: 10-100 Гц, плотность энергии: 0.45-\SI{0.14}{\mega\joule\per\meter\squared}) 
  наряду со стационарными плазменными потоками (плотность мощности: 1-\SI{10}{\mega\watt\per\meter\squared}) ведет к снижению скорости накопления дейтерия в 
  вольфраме при длительности облучения более \SI{10}{с} по сравнению со случаем стационарного облучения; 
  \item Эффективность (доля вышедших частиц) выхода водорода из поверхностных слоев вольфрама под действием импульсных 
  (\SI{10}{\nano\second} "--- \SI{5}{\milli\second}) тепловых нагрузок увеличивается с длительностью теплового воздействия. Поверхностные процессы снижают эффективность 
  выхода водорода из слоев вольфрама при длительности теплового воздействия менее \SI{10}{\micro\second};
  \item Оценка атомарной фракции в потоке десорбированного водорода при импульсных 
  тепловых нагрузках с миллисекундной ($\sim$1~\%) и наносекундной длительностью ($\sim$10~\%).
\end{enumerate}

{\reliability} полученных результатов обеспечивается \ldots \ Результаты находятся в соответствии с результатами, полученными другими авторами.

{\probation}
Основные результаты работы докладывались и обсуждались на российских и международных конференциях:
\begin{itemize}
    \item XXV "--- XXVIII конференции <<Взаимодействие плазмы с поверхностью>> (Москва, 2022 "--- 2025 гг.)
    \item Пятнадцатая международная школа молодых ученых и специалистов им. А.\,А. Курдюмова (Окуловка, 2022 г.)
    \item 26th International Conference on Plasma Surface Interaction in Controlled Fusion Devices (PSI-26, Marseille, France, 2024 г.)
\end{itemize}

Полученные результаты также представлялись и обсуждались на собраниях разработчиков программного пакета FESTIM.

{\contribution} Автор принимал активное участие \ldots

\ifnumequal{\value{bibliosel}}{0}
{%%% Встроенная реализация с загрузкой файла через движок bibtex8. (При желании, внутри можно использовать обычные ссылки, наподобие `\cite{vakbib1,vakbib2}`).
    {\publications} Основные результаты по теме диссертации изложены
    в~XX~печатных изданиях,
    X из которых изданы в журналах, рекомендованных ВАК,
    X "--- в тезисах докладов.
}%
{%%% Реализация пакетом biblatex через движок biber
    \begin{refsection}[bl-author, bl-registered]
        % Это refsection=1.
        % Процитированные здесь работы:
        %  * подсчитываются, для автоматического составления фразы "Основные результаты ..."
        %  * попадают в авторскую библиографию, при usefootcite==0 и стиле `\insertbiblioauthor` или `\insertbiblioauthorgrouped`
        %  * нумеруются там в зависимости от порядка команд `\printbibliography` в этом разделе.
        %  * при использовании `\insertbiblioauthorgrouped`, порядок команд `\printbibliography` в нём должен быть тем же (см. biblio/biblatex.tex)
        %
        % Невидимый библиографический список для подсчёта количества публикаций:
        \phantom{\printbibliography[heading=nobibheading, section=1, env=countauthorvak,          keyword=biblioauthorvak]%
        \printbibliography[heading=nobibheading, section=1, env=countauthorwos,          keyword=biblioauthorwos]%
        \printbibliography[heading=nobibheading, section=1, env=countauthorscopus,       keyword=biblioauthorscopus]%
        \printbibliography[heading=nobibheading, section=1, env=countauthorconf,         keyword=biblioauthorconf]%
        \printbibliography[heading=nobibheading, section=1, env=countauthorother,        keyword=biblioauthorother]%
        \printbibliography[heading=nobibheading, section=1, env=countregistered,         keyword=biblioregistered]%
        \printbibliography[heading=nobibheading, section=1, env=countauthorpatent,       keyword=biblioauthorpatent]%
        \printbibliography[heading=nobibheading, section=1, env=countauthorprogram,      keyword=biblioauthorprogram]%
        \printbibliography[heading=nobibheading, section=1, env=countauthor,             keyword=biblioauthor]%
        \printbibliography[heading=nobibheading, section=1, env=countauthorvakscopuswos, filter=vakscopuswos]%
        \printbibliography[heading=nobibheading, section=1, env=countauthorscopuswos,    filter=scopuswos]}%
        %
        \nocite{*}%
        %
        {\publications} Основные результаты по теме диссертации изложены в~\arabic{citeauthor}~печатных изданиях,
        \arabic{citeauthorvak} из которых изданы в журналах, рекомендованных ВАК%
        \ifnum \value{citeauthorscopuswos}>0%
            , \arabic{citeauthorscopuswos} "--- в~периодических научных журналах, индексируемых Web of~Science и Scopus%
        \fi%
        \ifnum \value{citeauthorconf}>0%
            , \arabic{citeauthorconf} "--- в~тезисах докладов.
        \else%
            .
        \fi%
        \ifnum \value{citeregistered}=1%
            \ifnum \value{citeauthorpatent}=1%
                Зарегистрирован \arabic{citeauthorpatent} патент.
            \fi%
            \ifnum \value{citeauthorprogram}=1%
                Зарегистрирована \arabic{citeauthorprogram} программа для ЭВМ.
            \fi%
        \fi%
        \ifnum \value{citeregistered}>1%
            Зарегистрированы\ %
            \ifnum \value{citeauthorpatent}>0%
            \formbytotal{citeauthorpatent}{патент}{}{а}{}%
            \ifnum \value{citeauthorprogram}=0 . \else \ и~\fi%
            \fi%
            \ifnum \value{citeauthorprogram}>0%
            \formbytotal{citeauthorprogram}{программ}{а}{ы}{} для ЭВМ.
            \fi%
        \fi%
        % К публикациям, в которых излагаются основные научные результаты диссертации на соискание учёной
        % степени, в рецензируемых изданиях приравниваются патенты на изобретения, патенты (свидетельства) на
        % полезную модель, патенты на промышленный образец, патенты на селекционные достижения, свидетельства
        % на программу для электронных вычислительных машин, базу данных, топологию интегральных микросхем,
        % зарегистрированные в установленном порядке.(в ред. Постановления Правительства РФ от 21.04.2016 N 335)
    \end{refsection}%
    \begin{refsection}[bl-author, bl-registered]
        % Это refsection=2.
        % Процитированные здесь работы:
        %  * попадают в авторскую библиографию, при usefootcite==0 и стиле `\insertbiblioauthorimportant`.
        %  * ни на что не влияют в противном случае
    \end{refsection}%
        %
        % Всё, что вне этих двух refsection, это refsection=0,
        %  * для диссертации - это нормальные ссылки, попадающие в обычную библиографию
        %  * для автореферата:
        %     * при usefootcite==0, ссылка корректно сработает только для источника из `external.bib`. Для своих работ --- напечатает "[0]" (и даже Warning не вылезет).
        %     * при usefootcite==1, ссылка сработает нормально. В авторской библиографии будут только процитированные в refsection=0 работы.
}

При использовании пакета \verb!biblatex! будут подсчитаны все работы, добавленные
в файл \verb!biblio/author.bib!. Для правильного подсчёта работ в~различных
системах цитирования требуется использовать поля:
\begin{itemize}
        \item \texttt{authorvak} если публикация индексирована ВАК,
        \item \texttt{authorscopus} если публикация индексирована Scopus,
        \item \texttt{authorwos} если публикация индексирована Web of Science,
        \item \texttt{authorconf} для докладов конференций,
        \item \texttt{authorpatent} для патентов,
        \item \texttt{authorprogram} для зарегистрированных программ для ЭВМ,
        \item \texttt{authorother} для других публикаций.
\end{itemize}
Для подсчёта используются счётчики:
\begin{itemize}
        \item \texttt{citeauthorvak} для работ, индексируемых ВАК,
        \item \texttt{citeauthorscopus} для работ, индексируемых Scopus,
        \item \texttt{citeauthorwos} для работ, индексируемых Web of Science,
        \item \texttt{citeauthorvakscopuswos} для работ, индексируемых одной из трёх баз,
        \item \texttt{citeauthorscopuswos} для работ, индексируемых Scopus или Web of~Science,
        \item \texttt{citeauthorconf} для докладов на конференциях,
        \item \texttt{citeauthorother} для остальных работ,
        \item \texttt{citeauthorpatent} для патентов,
        \item \texttt{citeauthorprogram} для зарегистрированных программ для ЭВМ,
        \item \texttt{citeauthor} для суммарного количества работ.
\end{itemize}
% Счётчик \texttt{citeexternal} используется для подсчёта процитированных публикаций;
% \texttt{citeregistered} "--- для подсчёта суммарного количества патентов и программ для ЭВМ.

Для добавления в список публикаций автора работ, которые не были процитированы в
автореферате, требуется их~перечислить с использованием команды \verb!\nocite! в
\verb!Synopsis/content.tex!.
