
{\actuality} Одним из перспективных направлений формирования новых источников энергии является управляемый термоядерный синтез (УТС). 
Наибольшие успехи на пути к практической реализации УТС достигнуты в установках с магнитным удержанием горячей плазмы типа токамак. 
В настоящее идет активная фаза строительства международного экспериментального реактора ИТЭР, спроектированного для демонстрации 
возможности квазистационарного (400 с) удержания термоядерной плазмы, а также введены в эксплуатацию наибольший в России токамак "--- Т15-МД, 
и наибольший в мире токамак JT60-SA, расположенный в Японии. Помимо этого, во множестве стран разрабатываются проекты установок следующего 
поколения для отработки реакторных технологий, в том числе в России ведется активное проектирование токамака ТРТ. Реализация термоядерной реакции 
планируется на основе дейтерий-тритиевой смеси. Использование радиоактивного трития накладывает определенные ограничения на эксплуатацию установки 
с точки зрения радиационной безопасности. В связи с этим, одной из важнейших задач будущих термоядерных установок является систематический контроль 
за накоплением трития в обращенных к плазме элементах (ОПЭ). 

В качестве основного материала ОПЭ рассматривается вольфрам. Согласно оценкам~\cite{Roth1}, захват трития в вольфрамовые элементы не будет определяющим 
процессом глобального удержания во время нормальных режимов работы установки. Однако в режимах с улучшенным удержанием горячей плазмы (H-мода) протекают 
переходные процессы, например ELM-неустойчивости (Edge Localised Modes), приводящие к периодическим импульсным потокам высокой мощности на поверхность ОПЭ. 
Результаты экспериментов по имитации воздействия мощных плазменных потоков в линейной плазменной установке КСПУ-Т~\cite{Ogorodnikova} показывают, что скорость 
накопления изотопов водорода может оказаться выше, чем в случае стационарного облучения, характерного для нормальных плазменных разрядов. Однако параметры 
облучения в установках такого типа не могут в полной мере воспроизвести условия, соответствующие крупным токамакам. Ввиду этого, исследование закономерностей 
накопления и удержания изотопов водорода под действием мощных импульсно-периодических плазменных нагрузок представляет повышенный интерес.
 
Схожей задачей является дистанционная диагностика содержания изотопов водорода в ОПЭ при помощи лазерно-индуцированной десорбции (ЛИД). Метод ЛИД 
заключается в нагреве участка исследуемой поверхности лазерным импульсом с последующим анализом состава вышедшего газа. Данная диагностика апробируется 
на сферическом токамаке Глобус-М2. Помимо этого, возможность применения ЛИД является одной из приоритетных задач исследований ИТЭР~\cite{loarte2020required}, 
а также рассматривается для российского проекта ТРТ~\cite{Razdobarin2022}.

{\aim} диссертационной работы является выявление закономерностей удержания изотопов водорода в вольфраме под действием импульсных плазменных и тепловых нагрузок.

Для~достижения поставленной цели необходимо было решить следующие~{\tasks}:
\begin{enumerate}[beginpenalty=10000] % https://tex.stackexchange.com/a/476052/104425
    \item Разработать и \fixme{валидировать} численную модель, описывающую транспорт изотопов водорода в металлах 
    под действием импульсных тепловых и плазменных нагрузок.
    \item Исследовать влияние быстрых переходных процессов, соответствующих ELM-неустойчивости в токамаках, на интегральное накопление 
    изотопов водорода в вольфрамовых ОПЭ. 
    \item \fixme{Проанализировать динамику изменения коэффициента рециклинга изотопов водорода во время быстрых переходных процессов, 
    соответствующих ELM-неустойчивости в токамаках.}
    \item Исследовать влияние поверхностных процессов на выход изотопов водорода из вольфрама при лазерном нагреве. 
    \item \fixme{Определить зависимость доли вышедших атомов изотопов водорода из поверхностных слоев вольфрама от параметров
    лазерного нагрева и теплофизических свойств материала.}
\end{enumerate}


{\novelty}
\begin{enumerate}[beginpenalty=10000] % https://tex.stackexchange.com/a/476052/104425
  \item Впервые \ldots
  \item Впервые \ldots
  \item Было выполнено оригинальное исследование \ldots
\end{enumerate}

{\influence} \ldots

{\methods} Достижение поставленной цели и решение сопутствующих задач осуществлялось путем проведения численного моделирования, 
позволяющего оценить влияние импульсных нагрузок в широком диапазоне параметров, обычно недоступном в рамках действующих экспериментальных 
и лабораторных установок. В качестве основного численного метода анализа транспорта изотопов водорода в вольфраме применялся 
метод конечных элементов, реализованный в программном пакете \href{https://github.com/festim-dev/FESTIM}{FESTIM}. Для демонстрации надежности и корректности 
разработанных моделей проводилась их верификация и валидация путем сравнения с экспериментальными результатами, представленными в литературе или полученными 
в рамках данной диссертационной работы. Построение аналитической модели, описывающей распределение изотопов водорода в вольфраме при наличии градиента 
температур (эффект Соре) и ловушек водорода, проводилось путем решения системы дифференциальных уравнений методом функции Грина.

{\defpositions}
\begin{enumerate}[beginpenalty=10000] % https://tex.stackexchange.com/a/476052/104425
  \item Одномерная аналитическая модель, описывающая стационарное распределение водорода при учете наличия градиента температур (эффект Соре)
  и центров захвата водорода в приближении мгновенной рекомбинации атомов водорода на обращенной к плазме поверхности и мгновенной рекомбинации 
  или нулевого потока атомов водорода на обратной поверхности и позволяющая прогнозировать предельное накопление изотопов водорода в обращенных 
  к плазме материалах. 
  \item Возникновение импульсно-периодических плазменных нагрузок, соответствующих ELM-событиям в токамаках (частота: 10 "--- 100 Гц, 
  длительность: $\sim$\SI{1}{\milli\second}, плотность энергии: 0.45 "--- \SI{0.14}{\mega\joule\per\meter\squared}), наряду со стационарными плазменными потоками 
  (плотность мощности: 1 "--- \SI{10}{\mega\watt\per\meter\squared}) ведет к снижению скорости накопления дейтерия в 
  вольфраме при длительности облучения более \SI{10}{\second} за счет дополнительного нагрева материала относительно случая облучения стационарными потоками плазмы.
  \item \fixme{Коэффициент рециклинга}
  \item Атомарная фракции в потоке водорода, десорбированного с поверхности вольфрама, растет с увеличением температуры поверхности и 
  уменьшением потока десорбированных частиц. Величина атомарной фракции в потоке десорбированного водорода может достигать $\sim$1~\% и $\sim$10~\% 
  при импульсном лазерном нагреве с наносекундной и миллисекундной длительностью до температуры плавления вольфрама.
  \item Поверхностные процессы снижают долю десорбированного водорода с чистой поверхности вольфрама при импульсном лазерном нагреве с длительностью 
  менее \SI{10}{\micro\second}.
\end{enumerate}

{\reliability} полученных результатов обеспечивается применением общепризнанного численного метода решения систем дифференциальных уравнений в 
частных производных, имплементированного в верифицированном и валидированном программном пакете FESTIM. Разработанные модели были валидированы путем 
сравнения результатов численных расчетов с экспериментальными данными, приведенными в литературе и полученными в рамках
данной диссертационной работы. Полученные результаты демонстрируют качественное и количественное согласие с 
литературными данными, полученными независимыми авторами на основе моделирования или экспериментального анализа.

{\probation}
Основные результаты работы докладывались и обсуждались на российских и международных конференциях:
\begin{itemize}
    \item XXV, XXVI, XXVII, XXVIII конференции <<Взаимодействие плазмы с поверхностью>> (Москва, 2022 "--- 2025 гг.)
    \item Пятнадцатая международная школа молодых ученых и специалистов им. А.\,А. Курдюмова (Окуловка, 2022 г.)
    \item 26th International Conference on Plasma Surface Interaction in Controlled Fusion Devices (PSI-26, Marseille, France, 2024 г.)
    \item 1st Open Source Software for Fusion Energy Conference (OSSFE, 2025 г.) 
\end{itemize}

Полученные результаты также представлялись и обсуждались на собраниях разработчиков программного пакета FESTIM.

{\contribution} Все результаты, выносимые на защиту, были получены автором или при его непосредственном участии. Лично автором были разработаны 
численные и аналитические модели, использованные для исследования закономерностей накопления и выхода изотопов водорода из вольфрама под действием 
импульсных плазменных и лазерных нагрузок. Постановка задач, выбор входных параметров для моделирования и анализ полученных результатов 
обсуждались с непосредственным научным руководителем д.ф.-м.н. \supervisorFioShort. Имплементация модели, учитывающей поверхностные 
процессы, в коде FESTIM \fixme{проводилась совместно с главным разработчиком кода Р. Делапорте-Матюран (Массачусетский технологический институт, США) при определяющем
участии автора, реализовавшим модель и проведшим ее верификацию и валидацию.} Эксперименты по ЛИД дейтерия из пленок вольфрама, со-осажденных вместе с дейтерием,
были проведены \fixme{коллективом ФТИ им. А.Ф. Иоффе в лице \dots} при непосредственном участии автора в постановке экспериментов, обработке результатов измерений 
и проведении сравнения с модельными данными.

\ifnumequal{\value{bibliosel}}{0}
{%%% Встроенная реализация с загрузкой файла через движок bibtex8. (При желании, внутри можно использовать обычные ссылки, наподобие `\cite{vakbib1,vakbib2}`).
    {\publications} Основные результаты по теме диссертации изложены
    в~XX~печатных изданиях,
    X из которых изданы в журналах, рекомендованных ВАК,
    X "--- в тезисах докладов.
}%
{%%% Реализация пакетом biblatex через движок biber
    \begin{refsection}[bl-author, bl-registered]
        % Это refsection=1.
        % Процитированные здесь работы:
        %  * подсчитываются, для автоматического составления фразы "Основные результаты ..."
        %  * попадают в авторскую библиографию, при usefootcite==0 и стиле `\insertbiblioauthor` или `\insertbiblioauthorgrouped`
        %  * нумеруются там в зависимости от порядка команд `\printbibliography` в этом разделе.
        %  * при использовании `\insertbiblioauthorgrouped`, порядок команд `\printbibliography` в нём должен быть тем же (см. biblio/biblatex.tex)
        %
        % Невидимый библиографический список для подсчёта количества публикаций:
        \phantom{\printbibliography[heading=nobibheading, section=1, env=countauthorvak,          keyword=biblioauthorvak]%
        \printbibliography[heading=nobibheading, section=1, env=countauthorwos,          keyword=biblioauthorwos]%
        \printbibliography[heading=nobibheading, section=1, env=countauthorscopus,       keyword=biblioauthorscopus]%
        \printbibliography[heading=nobibheading, section=1, env=countauthorconf,         keyword=biblioauthorconf]%
        \printbibliography[heading=nobibheading, section=1, env=countauthorother,        keyword=biblioauthorother]%
        \printbibliography[heading=nobibheading, section=1, env=countregistered,         keyword=biblioregistered]%
        \printbibliography[heading=nobibheading, section=1, env=countauthorpatent,       keyword=biblioauthorpatent]%
        \printbibliography[heading=nobibheading, section=1, env=countauthorprogram,      keyword=biblioauthorprogram]%
        \printbibliography[heading=nobibheading, section=1, env=countauthor,             keyword=biblioauthor]%
        \printbibliography[heading=nobibheading, section=1, env=countauthorvakscopuswos, filter=vakscopuswos]%
        \printbibliography[heading=nobibheading, section=1, env=countauthorscopuswos,    filter=scopuswos]}%
        %
        \nocite{*}%
        %
        {\publications} \textcolor{red}{Основные результаты по теме диссертации изложены в~\arabic{citeauthor}~печатных изданиях,
        \arabic{citeauthorvak} из которых изданы в журналах, рекомендованных ВАК}%
        \ifnum \value{citeauthorscopuswos}>0%
            , \arabic{citeauthorscopuswos} "--- в~периодических научных журналах, индексируемых Web of~Science и Scopus%
        \fi%
        \ifnum \value{citeauthorconf}>0%
            , \arabic{citeauthorconf} "--- в~тезисах докладов.
        \else%
            .
        \fi%
        \ifnum \value{citeregistered}=1%
            \ifnum \value{citeauthorpatent}=1%
                Зарегистрирован \arabic{citeauthorpatent} патент.
            \fi%
            \ifnum \value{citeauthorprogram}=1%
                Зарегистрирована \arabic{citeauthorprogram} программа для ЭВМ.
            \fi%
        \fi%
        \ifnum \value{citeregistered}>1%
            Зарегистрированы\ %
            \ifnum \value{citeauthorpatent}>0%
            \formbytotal{citeauthorpatent}{патент}{}{а}{}%
            \ifnum \value{citeauthorprogram}=0 . \else \ и~\fi%
            \fi%
            \ifnum \value{citeauthorprogram}>0%
            \formbytotal{citeauthorprogram}{программ}{а}{ы}{} для ЭВМ.
            \fi%
        \fi%
        % К публикациям, в которых излагаются основные научные результаты диссертации на соискание учёной
        % степени, в рецензируемых изданиях приравниваются патенты на изобретения, патенты (свидетельства) на
        % полезную модель, патенты на промышленный образец, патенты на селекционные достижения, свидетельства
        % на программу для электронных вычислительных машин, базу данных, топологию интегральных микросхем,
        % зарегистрированные в установленном порядке.(в ред. Постановления Правительства РФ от 21.04.2016 N 335)
    \end{refsection}%
    \begin{refsection}[bl-author, bl-registered]
        % Это refsection=2.
        % Процитированные здесь работы:
        %  * попадают в авторскую библиографию, при usefootcite==0 и стиле `\insertbiblioauthorimportant`.
        %  * ни на что не влияют в противном случае
    \end{refsection}%
        %
        % Всё, что вне этих двух refsection, это refsection=0,
        %  * для диссертации - это нормальные ссылки, попадающие в обычную библиографию
        %  * для автореферата:
        %     * при usefootcite==0, ссылка корректно сработает только для источника из `external.bib`. Для своих работ --- напечатает "[0]" (и даже Warning не вылезет).
        %     * при usefootcite==1, ссылка сработает нормально. В авторской библиографии будут только процитированные в refsection=0 работы.
}


