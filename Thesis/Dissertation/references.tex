\clearpage                                  % В том числе гарантирует, что список литературы в оглавлении будет с правильным номером страницы
%\hypersetup{ urlcolor=black }               % Ссылки делаем чёрными
%\providecommand*{\BibDash}{}                % В стилях ugost2008 отключаем использование тире как разделителя
\urlstyle{rm}                               % ссылки URL обычным шрифтом
\ifdefmacro{\microtypesetup}{\microtypesetup{protrusion=false}}{} % не рекомендуется применять пакет микротипографики к автоматически генерируемому списку литературы
% \insertbibliofull                           % Подключаем Bib-базы: все статьи единым списком
% Режим с подсписками
\begin{refcontext}[labelprefix=A]
    \insertbiblioauthor
\end{refcontext}                        % Подключаем Bib-базы: работы автора единым списком 
\insertbiblioexternal                      % Подключаем Bib-базы: статьи, не являющиеся статьями автора по теме диссертации
% Для вывода выберите и расскомментируйте одно из двух

%\insertbiblioauthorgrouped                 % Подключаем Bib-базы: работы автора сгруппированные (ВАК, WoS, Scopus и т.д.)
\ifdefmacro{\microtypesetup}{\microtypesetup{protrusion=true}}{}
\urlstyle{tt}                               % возвращаем установки шрифта ссылок URL
%\hypersetup{ urlcolor={urlcolor} }          % Восстанавливаем цвет ссылок
