\chapter{Обзор исследований накопления изотопов водорода в материалах ОПЭ}\label{ch:ch1}

\nomenclature[A, 0]{УТС}{Управляемый термоядерный синтез}
\nomenclature[A, 1]{ТЯУ}{Термоядерная установка}
\nomenclature[A, 2]{ИТЭР}{Международный экспериментынй термоядерный реактор}
\nomenclature[A, 3]{ОПЭ}{Обращенные к плазме элементы}
\nomenclature[A, 4]{DT-плазма}{Дейтерий-тритиевая плазма}
\nomenclature[A, 5]{H"=мода}{Режим с высоким временем удержанием энергии в плазме токамака (High confinement regime)}
\nomenclature[A, 6]{ЛИД}{Лазерно-инудуцированная десорбция}

Исследования в области управляемого термоядерного синтеза (УТС) с магнитным удержанием плазмы достигли важного этапа, на котором продемонстрировано длительное ($\sim100-\SI{1000}{\second}$) удержание энергии в горячей плазме. Однако инженерно-техническая реализация термоядерных установок (ТЯУ) промышленного масштаба потребует решения совокупности взаимосвязанных задач для осуществления квазистационарного режима горения плазмы с параметрами, необходимыми для протекания реакции термоядерного синтеза. Важным аспектом продолжительной работы установок такого типа являются взаимодействие пристеночной плазмы с поверхностью обращенных к плазме элементов (ОПЭ)~\cite{Krieger2025}. Эта взаимодействие существенно влияет на параметры разряда и во многом определяет срок службы ОПЭ, принимающих на себя основной поток частиц и энергии из плазмы, что накладывает ограничения на выбор материалов облицовки. Сопутствующими процессами данного взаимодействия являются накопление и обратное газовыделение изотопов водорода "--- рециклинг, контроль которого оказывается важным при осуществлении длительных разрядов. Немаловажным аспектом также является мониторинг содержания радиоактивного трития, накопление которого в ОПЭ должно быть минимизировано. Исследование закономерностей захвата и обратного газовыделения изотопов водорода из ОПЭ таким образом представляет собой ключевую задачу будущих ТЯУ.

\section{Плазменные нагрузки на ОПЭ в ТЯУ}\label{sec:ch1/sec1}

% Тепловые нагрузки
Обращенные к плазме элементы в ТЯУ на базе токамака будут подвержены воздействию интенсивных потоков тепла и частиц. В действующих установках среднее значение плотности потока тепла (отношение полной мощности, попадающей на стенку, к ее проективной площади) на ОПЭ оценивается на уровне \SIrange{0.2}{0.6}{\mega\watt\per\metre\squared}~\cite{Mazul2021}. Особенности удержания плазмы в магнитной конфигурации токамака однако приводят к появлению пространственно-временного распределения этой нагрузки. Диверторная область токамаков является наиболее нагруженной. Прогностическое моделирование для строящегося токамака ИТЭР показывает, что стационарные нагрузки в нормальных разрядах\footnote{Под <<нормальными разрядами>> подразумеваются квазистационарные режимы горения плазмы, не подверженных влиянию глобальных неустойчивостей} могут достигать величины \SIrange{5}{15}{\mega\watt\per\meter\squared}~\cite{Pitts2019,Orrico2023} вблизи пересечения сеператрисы и внешних панелей дивертора, приводя к нагреву поверхности до температуры в \SIrange{500}{1000}{\kelvin}.

Величина приходящих на ОПЭ потоков тепла может меняться в ходе различных переходных процессов, которые можно разделить на медленные и быстрые. К медленным можно отнести неизбежные процессы зажигания и затухания разряда длительностью порядка \SI{10}{\second} для ИТЭР, а также допустимые кратковременные повышения нагрузок до уровня \SI{20}{\mega\watt\per\meter\squared} длительностью в несколько секунд~\cite{Pitts2017}. Инициирование быстрых переходных процессов связано с развитием неустойчивостей разного рода~\cite{hender2007mhd}. На рисунке~\cref{fig:ch1/Heat_loads_diagram} приведена сравнительная диаграмма тепловых потоков, приходящих на диверторные пластины ИТЭР~\cite{Linke2019}.

\begin{figure}[ht]
    \centerfloat{
        \includegraphics[scale=1]{Heat_loads_diagram.png}
    }
    \caption{Параметры ожидаемых тепловых потоков, приходящих на ОПЭ в ИТЭР~\cite{Linke2019}. Закрашенная область соответствует области нанесения необратимого ущерба поверхности}\label{fig:ch1/Heat_loads_diagram}
\end{figure}

Значительную опасность представляют три типа событий: большие срывы тока, вертикальные смещения плазменного шнура и импульсно-периодические плазменные нагрузки во время ELM"=событий. Большие срывы тока происходят при развитии глобальных магнитогидродинамических неустойчивостей. Они представляют наибольшую опасность, так как могут приводить к выбросу значительной части запасенной в плазме энергии на ОПЭ за времена порядка \SIrange{10}{100}{\milli\second}. Вертикальные смещения характерны для D-образного сечения плазменного шнура при утере устойчивости в вертикальном направлении за счет быстрых изменений параметров плазмы. События неконтролируемого вертикального смещения ведут к выходу плазмы за сепаратрису, создавая дополнительную нагрузку на ОПЭ за время порядка \SIrange{0.1}{1}{\second}. 

Возникновение импульсно-периодических потоков высокоэнергетичных частиц на ОПЭ типично для плазменных разрядов в H"=моде, переход в которую сопровождается формированием транспортного барьера на периферии плазмы. Стабильность транспортного барьера может быть нарушена при росте градиентов тока и/или давления плазмы на периферии до тех пор, пока не будет достигнут предел магнитогидродинамической устойчивости. В результате возникающие неустойчивости (ELM"=неустойчивости) вызывают быструю релаксацию давления с выбросом плазменного филамента в периферийную область плазмы (скреп-слой). Распространяясь вдоль силовых магнитных линий в скреп-слое, пламенные филаменты в итоге приходят на ОПЭ, вызывая мощное кратковременное (\( \leq \SI{1}{\milli\second} \)) облучение поверхности интенсивными потоками тепла и высокоэнергетичных частиц. Выделяют несколько типов ELM-событий, дифференцируемых по ряду признаков: режим плазменного разряда, доля уносимой энергии из плазмы, частота повторения, наличие предшествующей неустойчивости (прекурсора). Наибольшую угрозу представляют ELM-события первого типа, которые приводят к потерям энергии из плазмы на уровне \SIrange{1}{10}{\percent} c характерной частотой \SIrange{1}{100}{\hertz}. Выделение основной части унесенной энергии во время ELM-событий происходит в области дивертора, однако также возможен вынос энергии и на элементы первой стенки за счет радиального транспорта во время распространения вдоль силовых линий~\cite{zohm1996edge,Krieger2025,Leonard2014}.

Согласно оценкам~\cite{Loarte2003, hender2007mhd, Pitts2017, Pitts2019}, потоки тепла во время переходных процессов в ИТЭР могут достигать $\sim\SI{10}{\giga\watt\per\meter\squared}$. Нагрузки такого уровня могут нанести серьезный ущерб поверхности (растрескивание, плавление, испарение и т.д.). Предотвращение возможного ущерба является приоритетной задачей, для решения которой разрабатываются соответствующие операционные системы и подходы~\cite{Lang2013,Evans2013,Lehnen2015}. Ввиду этого, далее в работе основное внимание будет уделено переходным процессам типа ELM"=событий, последствия развития которых предполагаются либо допустимыми, либо могут быть минимизированы до приемлемого уровня.

Потоки частиц, приходящие на ОПЭ в действующих установках, в основном состоят из электронов, ионов и нейтралов перезарядки изотопов водорода. Возможно также наличие малой доли более тяжелых частиц материалов ОПЭ, остаточного газа или примесей, вводимых в установку для кондиционирования стенок или распределения тепловых нагрузок. В проектируемых ТЯУ, в том числе и токамаке ИТЭР, корпускулярные нагрузки также будут включать продукты DT-реакции синтеза: высокоэнергетичные нейтроны (\SI{14.1}{\mega\electronvolt}) и альфа-частицы (\SI{3.5}{\mega\electronvolt}). Длительные режимы работы установок будут сопровождаться высокой дозой облучения ОПЭ, что также может приводить к модификации как поверхностной, так и объемной структуры. К тому же, важной задачей с точки зрения обеспечения радиационной безопасности является минимизация накопления радиоактивного трития в элементах установки на протяжении ее срока эксплуатации.

Плотность потока частиц на элементы первой стенки ИТЭР оценивается на уровне \SIrange{e18}{e21}{\per\metre\squared\per\second}~\cite{DeTemmerman2021, Rivals2022}, когда для диверторных пластин "--- на уровне \SIrange{e23}{e24}{\per\metre\squared\per\second}~\cite{Pitts2019,Orrico2023}. В зависимости от параметров и режима разряда энергия приходящих ионов может достигать величины порядка \SI{10}{\electronvolt} для первой стенки и порядка \SI{100}{\electronvolt} для области дивертора. На элементы первой стенки возможен также приход высокоэнергетичных нейтралов с энергией порядка килоэлектронвольт, образованных в результате резонансной перезарядки. Оценка суммарной дозы облучения пластин дивертора в ИТЭР показывает, что за 5000 номинальных разрядов длительностью \SI{400}{\second} ее величина составит \( \sim \SIrange{e30}{e31}{\per\metre\squared} \). Как продемонстрировано на сравнительных диаграммах на рисунке~\cref{fig:ch1/fluxes_comparison}, условия эксплуатации ОПЭ в ИТЭР будут намного суровее, чем в действующих токамаках.
\begin{figure}[ht]
    \centerfloat{
        \includegraphics[scale=0.9]{fluxes_comparison.png}
    }
    \caption{Сравнение оцененной ионной дозы после 5000 тысяч нормальных плазменных разрядов в различных токамаках (слева) и характерных условий облучения в различных типах экспериментов (справа)~\cite{DeTemmerman2018}}\label{fig:ch1/fluxes_comparison}
\end{figure}
Энергии ионов в диверторе сопоставимы с энергиями, используемыми в различных методах плазменного осаждения и травления, однако плотность потока ионов на несколько порядков выше. Помимо этого, средние значения плотности потока и энергии ионов, приходящих на поверхность ОПЭ, могут увеличиваться во время переходных процессов. Анализ данных с диагностических систем (зонды Ленгмюра и \(\mathrm{D}_\alpha\)-спектроскопия) токамака JET~\cite{Guillemaut2015,Guillemaut2018} демонстрирует кратное увеличение измерительных сигналов во время ELM"=событий, что соответствуют пропорциональному росту потока частиц. Считается, что энергия частиц во время ELM"=событий пропорциональна температуре плазмы на пьедестале~\cite{Eich2017}. При сохранении аналогичных зависимостей для токамака масштаба ИТЭР можно ожидать соответствующее превышение уровня стационарных потоков в несколько раз с энергией приходящих ионов порядка килоэлектронвольт.

Немаловажным остается влияние образования гелия и нейтронов на эксплуатацию ОПЭ. Генерируемые в ходе DT-реакции частицы гелия должны терять большую часть своей энергии в центре плазменного шнура, однако облучение материалов ионами гелия с низкой энергией также может оказывать существенное влияние. Известно, что облучение тугоплавких металлов ионами гелиевой плазмы индуцирует формирование гелиевых пузырей в объеме материала, а также может приводить к его структурным изменениям с образованием высокопостристых структур~\cite{Ueda2018,Kajita2018,Fedorovich2019}. Любые незапланированные структурные изменения морфологии поверхности ОПЭ будут влиять на процессы взаимодействия плазмы с ней, что может приводить к глобальным изменениям в режимах удержания плазмы.

Взаимодействие ионов с ОПЭ происходит по большей части вблизи поверхности. Падающий на поверхность поток частиц может приводить либо к обратной эмиссии частиц (отражение, распыление), либо к их имплантации. Средняя глубина внедрения частиц может достигать нескольких десятков нанометров в зависимости от состава и структуры мишени, сорта приходящих частиц и их энергии~\cite{eckstein2010penetration}. Ввиду гораздо большей глубины пробега нейтронов основные процессы их взаимодействия происходят в объеме материалов. Облучение нейтронными потоками ведет к объемному нагреву материалов, трансмутации элементов, а также образованию дефектов кристаллической решетки за счет развития каскадов столкновений. Прогнозируемый уровень повреждений в конце срока службы токамака ИТЭР оценивается гораздо ниже инженерных ограничений~\cite{Villari2013}. Тем не менее, образование радиационных дефектов в материалах ОПЭ в ходе работы установки может влиять на удержание изотопов водорода.

\section{Материалы ОПЭ в ТЯУ}\label{sec:ch1/sec2}

Влияние описанных в предыдущим разделе отдельных характерных типов нагрузок широко исследовалось в лабораторных условиях (в меньшей степени исследованы эффектыы, связанные с облучением высокоэнергетичными нейтронами). В общем случае, они будут оказывать совместное влияние, что качественно проиллюстрировано на рисунке~\cref{fig:ch1/synergetic_diagram}. Синергетические эффекты воздействия на ОПЭ в условиях ТЯУ могут создавать новые вызовы для выбора их оптимальной конструкции.
\begin{figure}[ht]
    \centerfloat{
        \includegraphics[scale=0.5]{synergetic_diagram.png}
    }
    \caption{Характерные типы нагрузок и их основные последствия для ОПЭ при работе ТЯУ~\cite{Linke2019}}\label{fig:ch1/synergetic_diagram}
\end{figure}
Эти аспекты также накладывают совокупность требований и на выбор используемых материалов. К основным факторам, определяющим этот выбор, можно отнести термостойкость и теплопроводность, эрозионную устойчивость при ионном облучении, способность накапливать изотопы водорода, устойчивость и низкую активируемость при нейтронном облучении. Высокие термостойкость и теплопроводность, а также низкая эрозия при облучении легкими ионами характерны для тугоплавких металлов с большим зарядовым числом ядра ($Z$). Однако распыление и попадание в область горячей плазмы атомов таких материалов может приводить к увеличению потерь мощности на излучение из плазмы~\cite{Ptterich2019}. Легкие материалы подвержены большей эрозии поверхности и могут влиять на динамику накопления изотопов водорода в ТЯУ. Основными материалами, применяемыми в современных установках для облицовки ОПЭ, являются графит (JT-60SA~\cite{Shirai2024}, T-15МД~\cite{Velikhov2024}), бериллий (JET~\cite{Maggi2024,Kappatou2025}), молибден(EAST~\cite{Gong2024}) и вольфрам/вольфрамовые покрытия (JET, EAST, WEST~\cite{Shi2025}, ASDEX-Upgrade~\cite{Rohde2009}). \nomenclature[P, 01]{$Z$}{Атомный номер элемента}

Углеродные материалы являются привлекательным выбором для ОПЭ ввиду ряда преимуществ. Проникновение углерода в плазму ведет к малым потерям мощности с излучением из-за его низкого зарядового числа ($Z=6$). Углеродные материалы характеризуются высокими теплостойкостью и значением теплопроводности ($\SIrange{100}{300}{\watt\per\meter\squared\per\kelvin}$ при комнатной температуре~\cite{Merola2004, Begrambekov2023}), сопоставимой с металлами. Помимо этого, нагрев поверхности до высоких температур (>\SI{3800}{\kelvin}) ведет к сублимации материала, а не плавлению, что предотвращает вероятность развития процессов взаимодействия расплава с приповерхностной плазмой и магнитным полем. Однако углеродные материалы более подвержены распылению при облучении легкими ионами, чем, например, вольфрам. а также характеризуются высоким коэффициентом химического распыления при облучении изотопами водорода. Для сравнения на рисунке~\cref{fig:ch1/sputerring_yields} приведены расчетные коэффициенты распыления углерода (физического и химического), бериллия и вольфрама. Влияние химического распыления может быть минимизировано обеспечением <<благоприятного>> режима облучения, но другим важным недостатком углеродных материалов является накопление изотопов водорода. Изотопы водорода образуют прочные C-H связи, что может приводить к чрезмерному накоплению трития в переосажденных пленках и затруднять его извлечение из установки~\cite{Gasparyan2024}. Помимо этого, интенсивное облучение потоками нейтронов и компонентами плазмы может приводить к деградации теплофизических свойств~\cite{Wu1994} и структурным изменениям поверхностных слоев~\cite{Wang2018,Begrambekov2019,Seyedhabashi2025}.

\begin{figure}[ht]
    \centerfloat{
        \includegraphics[scale=1]{sputerring_yields.pdf}
    }
    \caption{Расчетные зависимости коэффициентов физического распыления бериллия, углерода и вольфрама от энергии некоторых типов ионов при нормальном падении~\cite{international2001iaea, behrisch_2025}. Для углерода также приведены зависимости коэффициента химического распыления дейтерием при комнатной температуре и различных значениях плотности потока~\cite{Roth1999,Roth2004}}\label{fig:ch1/sputerring_yields}
\end{figure}

Бериллий, как и углерод, обладает низким атомным номером ($Z=4$), что обеспечивает относительно хорошую совместимость с плазмой. Он также характеризуется высокой теплопроводностью (\( \approx \SI{200}{\watt\per\meter\per\K} \)) и относительно высокой температурой плавления (\SI{1560}{\kelvin})~\cite{Ho1974}. По сравнению с углеродными материалами, скорость химического распыления~\cite{Brezinsek2014} и накопления изотопов водорода~\cite{DeTemmerman2021} в нем существенно меньше. Переход к металлической облицовке в токамаке JET привел к снижению интегральной эрозии ОПЭ и, как следствие, накоплению изотопов водорода более чем на порядок~\cite{Brezinsek2015}. Бериллий также химически активен, что позволяет эффективно связывать кислород, уменьшая уровень примеси в плазме и формируя более устойчивый к распылению оксид. Несмотря на явные преимущества бериллия как материала для ОПЭ, он обладает рядом существенных недостатков. Первым из них является токсичность бериллиевой пыли, что требует применения специальных мер при работе с ним~\cite{Strupp2011}. Вторым недостатком является высокий коэффициент физического распыления легкими ионами по сравнению с графитом (рис.~\cref{fig:ch1/sputerring_yields}), что может сказываться на сроке службы ОПЭ под действием интенсивного ионного облучения. Следующим негативным свойством является деградация теплофизических свойств и охрупчивание при облучении нейтронами и гелием~\cite{Kesternich2003,Gilbert2012}. Совокупность этих и иных особенностей послужили причиной принятия решения об отказе от применение бериллия в качестве ОПЭ токамака ИТЭР~\cite{Barabaschi2025}, но использование бериллия пока что рассматривается для российского токамака ТРТ~\cite{Mazul2021,Piskarev2024}.

Молибден ($Z=42$) и вольфрам ($Z=74$) являются тугоплавкими металлами с высокими температурами плавления: \SI{2896}{\kelvin} и \SI{3695}{\kelvin}, соответственно. Однако вольфрам предпочтительнее по причине меньшей активируемости нейтронами, хоть и обладает рядом свойств усложняющих технологическое производство элементов облицовки с ним~\cite{Piskarev2024}. Исходя из этого, основное внимание будет уделено вольфраму. Помимо высокой теплостойкости, вольфрам характеризуется низким коэффициентом распыления легкими ионами (рис.~\cref{fig:ch1/sputerring_yields}), совместимостью с нейтронным облучением и низкой растворимостью изотопов водорода~\cite{Roth2011, Pintsuk2012,Rieth2019}. Явным недостатком вольфрама является вероятность распыления тяжелыми примесными ионами в плазме или ее легкими основными компонентами с высокой энергией, сопровождаемое загрязнением плазмы частицами с высоким атомным номером. Как отмечалось в разделе~\cref{sec:ch1/sec1}, потоки частиц плазмы с высокой энергией ожидаются в ходе различных переходных процессов. Учитывая риски появления тяжелой примеси в плазме при распылении вольфрама, рассматривается использование легких материалов (бор ($Z=3$), литий ($Z=5$)) для нанесения кондиционирующих покрытий~\cite{Winter1996,Wauters2020}. Известна также проблема образования трещин на поверхности вольфрама при циклических тепловых нагрузках, характерных для ELM"=событий. Дополнительно необходимо отметить активно исследуемые в последнее время изменения поверхности вольфрама при облучении ионами гелиевой плазмы. В условиях, характерных для диверторной области токамаков, облучение ионами гелия ведет к формированию высокопористой наноструктуризованной морфологии, называемой <<пух>>~\cite{Ueda2018,Kajita2018,Fedorovich2019}. Рост вольфрамового <<пуха>> на поверхности может повысить вероятность зажигания униполярных дуг с соответствующем увеличением уровня эрозии поверхности. Невзирая на недостатки вольфрама, он выбран в качестве основного материала ОПЭ для ИТЭР~\cite{Pitts2025} и находится в приоритетном списке как для ТРТ~\cite{Piskarev2024}, так и для проектируемых в настоящее время установок. 

\section{Механизмы длительного накопления изотопов водорода в ТЯУ}\label{sec:ch1/sec3}
Как показно в предыдущем разделе, вольфрам обладает совокупностью свойств, определяющей его применимость в качестве материала ОПЭ. Перспектива его использования инициировала всестороннее изучение закономерностей накопления изотопов водорода в условиях, ожидаемых в ТЯУ. Актуальность детального анализа во многом обусловлена необходимостью контроля за содержанием радиоактивного трития. Так, для токамака ИТЭР установлен административный лимит в \SI{1}{\kilogram} по интегральному накоплению в элементах вакуумной камеры с учетом возможных погрешностей и накопления в крионасосах~\cite{Roth1}.

Изотопы водорода могут накапливаться на поверхности ОПЭ за счет адсорбции низкоэнергетичных атомов или молекул. Совокупность каналов адсорбции является процессом динамического (краткосрочного) удержания и не представляет особой проблемы, так как дегазация поверхности протекает достаточно быстро при повышенных температурах. Выделяют два основных канала долгосрочного удержания изотопов водорода: захват в объеме материала и соосаждение~\cite{Gasparyan2024, Skinner2009}. Также можно отметить накопление изотопов водорода на поверхности или в объеме пылевых микрочастиц, образующихся в процессе эрозии поверхностных слоев ОПЭ. Накопление в пылевых частицах сильно зависит от механизма их образования и во многом может описываться процессами, определяющими захват водорода в объеме материалов или при соосаждении. К тому же, подтверждения существенного вклада в глобальное накопление от данного механизма обнаружено не было. В силу этого, внимание будет уделено двум другим процессам долгосрочного удержания изотопов водорода в ТЯУ.

Основные процессы, определяющие длительное накопление, схематически представлены на рисунке~\cref{fig:ch1/retention_mechanisms}.
\begin{figure}[ht]
    \centerfloat{
        \includegraphics[scale=1]{retention_mechanisms.png}
    }
    \caption{Схематическое представление процессов захвата изотопов водорода при имплантации и соосаждении. Синие круги соответствуют атомам водорода, серые "--- атомам облучаемого материала}\label{fig:ch1/retention_mechanisms}
\end{figure}
Захват изотопов водорода при ионном облучении изначально происходит за счет имплантации частиц. Вероятность внедрения ионов определяется интегральным коэффициентом отражения, зависящим от параметров облучения и сортов взаимодействующих частиц. При попадании в объем внедренные частицы продолжают движение, теряя свою начальную энергию за счет упругих и неупругих потерь, пока не термализуются. Подвижность атомов водорода в металлах, как вольфрам, остается достаточно высокой при повышенных температурах, что позволяет им диффундировать обратно к поверхности и десорбироваться за счет различных процессов, например ассоциативной рекомбинации. С другой стороны, характерная глубина внедрения ионов в условиях токамака не превышает десятков нанометров. Интенсивное облучение поверхности приводит к быстрому насыщению водорода в зоне имплантации, что инициирует распространение водорода вглубь материала. В ходе диффузии подвижные атомы водорода могут захватываться различными дефектами кристаллической решетки, представляющими собой более глубокие потенциальные ямы по сравнению с межузельными положениями. Учитывая диффузионный характер, величина интегрального накопления пропорциональна квадратному корню из времени (дозы) облучения, что наблюдается в многочисленных экспериментах~\cite{Ogorodnikova2003,Ogorodnikova2009,Sugiyama2014,Zhang2020}. Ввиду малой растворимости водорода в вольфраме, долгосрочное накопление будет определяться распределением и концентрацией дефектов, образование которых будет происходить при ионном и нейтронном облучении.

Процесс соосаждения водорода и материалов первой стенки в токамаках происходит в несколько итерационных этапов. Непрерывное распыление материалов ОПЭ приводит к попаданию примесных атомов в плазму, в которой они ионизуются. Ионы материалов стенки за счет процессов переноса в пристеночной плазме в итоге нейтрализуются и осаждаются на других элементах облицовки, инициируя рост пленок. В то же время осаждаемые пленки подвергаются непрерывному облучению потоками частиц из плазмы. Совместное протекание обоих процессов ведет к практически равномерному накоплению изотопов водорода, интегральная величина которого растет приблизительно линейно с временем.

Результаты, полученные на современных токамаках, указывают на то, что соосаждение является доминирующим каналом накопления изотопов водорода. Параметры удержания водорода, физические свойства и скорость роста пленок сильно зависят от условий совместного осаждения~\cite{Gasparyan2019,Krat2020,Krat2025}. Доля водорода в пленках может достигать десятков процентов для различных материалов (см. Рисунок~\cref{fig:ch1/codeposition_review}).
\begin{figure}[ht]
    \centerfloat{
        \includegraphics[scale=0.4]{codep_review.png}
    }
    \caption{Содержание изотопов водорода в зависимости от температуры совместного осаждения для пленок бора (голубая область), вольфрама (серая область), бериллия (зеленая область) и углерода (красная область)~\cite{Pitts2025}}\label{fig:ch1/codeposition_review}
\end{figure}
Примечательно, что содержание водорода оказывается систематически меньше в пленках вольфрама по сравнению с другими материалами. Помимо этого, скорость образования вольфрамовых пленок может быть меньше из-за более высокого порога распыления. Однако недавний \textit{post-mortem} анализ образцов-свидетелей из токамака WEST указывает на образование переосажденных слоев с толщиной несколько микрометров~\cite{Bucalossi2024}.

\section{Взаимодействие изотопов водорода с вольфрамом}\label{sec:ch1/sec4}

Транспорт внедренных атомов водорода в вольфраме включает множество механизмов, которые можно разделить на несколько групп: диффузионные механизмы, взаимодействие с различными типами дефектов и процессы на поверхности. Качественное описание каждой из групп можно дать на основе одномерного представления пространственного распределения потенциальной энергии атомов водорода вблизи поверхности материала (см. рисунок~\cref{fig:ch1/potential_diagram_all}). Важно заметить, что в реальности ситуация оказывается гораздо сложнее, так как распределение энергии водорода в материале с дефектами кристаллической решетки является трехмерным с совокупностью локальных экстремумов и седловых точек. Получение детальной информации о распределении потенциальной энергии возможно при помощи \textit{ab initio} методов, как теория функционала электронной плотности (DFT). \nomenclature[A, 7]{DFT}{Теория функционала электронной плотности (Density functional theory)}

\begin{figure}[ht]
    \centerfloat{
        \includegraphics[scale=1]{potential_diagram_all.pdf}
    }
    \caption{Схематическая диаграмма потенциальной энергии водорода вблизи поверхности металла с положительной теплотой растворения. Уровни энергии отсчитываются от связанного состояния атома в молекуле водорода, расположенной далеко в вакууме}\label{fig:ch1/potential_diagram_all}
\end{figure}

\subsection{Диффузия}

Отталкивающий характер взаимодействия между атомами кристаллической решетки и водорода определяет предпочтительную оккупацию последним межузельных положений, являющихся локальными минимумами потенциальной энергии. Для вольфрама с объемно-центрированной кубической решеткой наиболее устойчивыми положениями являются тетраэдрическое и октаэдрическое, причем первое является более энергетически выгодным со значением теплоты растворения $Q_\mathrm{s}$ равным \SIrange{0.9}{1.0}{\electronvolt}~\cite{Heinola2010,Johnson2010,Fernandez2015,Zhou2024}. В процессе колебаний атомы водорода могут перескакивать между равновесными положениями, распространяясь стохастически по объему материала. Направленный диффузионный поток атомов водорода $\mathbf{J}$ может быть обусловлен градиентами химического потенциала с вкладом $\mathbf{J}_c$ и температуры с вкладом $\mathbf{J}_T$~\cite{Longhurst1985, Krom1999, Martinez2021}:
\begin{equation}
    \mathbf{J}=\mathbf{J}_c+\mathbf{J}_T.
\end{equation}
В приближении малой концентрации растворенного водорода и отсутствия градиента напряжений в решетке вольфрама вклад от градиента химического потенциала можно редуцировать к влиянию градиента концентрации (закон Фика), определяющего поток $\mathbf{J}_{c}$:
\begin{equation}
    \mathbf{J}_{c} = -D \nabla \cm,
\end{equation}
где $D$ "--- коэффициент диффузии, \si{\meter\squared\per\second}; $\cm$ "--- концентрация атомов подвижного водорода, \si{\per\meter\cubed}. Диффузия, вызванная градиентом концентрации, является термоактивируемым процессом и протекает быстрее с ростом температуры. Температурная зависимость коэффициента диффузии обычно описывается в соответствии с законом Аррениуса:
\begin{equation}
    D(T)=D_0 \exp\left( -\frac{E_\mathrm{D}}{k_\mathrm{B}T} \right),
\end{equation}
где $E_\mathrm{D}$ "--- энергия активации диффузии, \si{\electronvolt}; $k_\mathrm{B}=\SI{8.617e-5}{\electronvolt\per\kelvin}$ "--- постоянная Больцмана; $T$ "--- абсолютная температура, \si{\kelvin}.
\nomenclature[P, 02]{$Q_\mathrm{s}$}{Теплота растворения, \si{\electronvolt}}
\nomenclature[P, 03]{$J$}{Плотность потока атомов, \si{\per\meter\squared\per\second}}
\nomenclature[P, 04]{$J_{c}$}{Плотность потока атомов, индуцированного градиентом концентрации, \si{\per\meter\squared\per\second}}
\nomenclature[P, 05]{$J_{T}$}{Плотность потока атомов, индуцированного градиентом температуры, \si{\per\meter\squared\per\second}}
\nomenclature[P, 06]{$D$}{Коэффициент диффузии, \si{\meter\squared\per\second}}
\nomenclature[P, 07]{$\cm$}{Объемная концентрация подвижных атомов, \si{\per\meter\cubed}}
\nomenclature[P, 08]{$E_\mathrm{D}$}{Энергия активация диффузии, \si{\electronvolt}}
\nomenclature[P, 09]{$k_\mathrm{B}$}{Постоянная Больцмана, \si{\electronvolt\per\kelvin}}
\nomenclature[P, 10]{$T$}{Абсолютная температура, \si{\kelvin}}

Определение коэффициента диффузии проводилось как посредством экспериментов, так и путем моделирования. Параметры коэффициента диффузии, полученные Фраунфельдером ($D_0=\SI{4.1e-7}{\metre\squared\per\second}$, $E_\mathrm{D}=\SI{0.39}{\electronvolt}$, $T=\SIrange{1100}{2400}{K}$)~\cite{frauenfelder1969solution}, продолжительное время считались наиболее надежными. Последующие результаты моделирования методом DFT~\cite{Heinola2010,Johnson2010,Fernandez2015,Zhou2024} и недавние эксперименты~\cite{Holzner2020} указывают на большую подвижность атомов водорода в вольфраме с энергией активации диффузии в диапазоне \SIrange{0.2}{0.28}{\electronvolt}. Указанный диапазон согласуется с результатами Фраунфельдера при учете экспериментальных данных только в высокотемпературной области, когда влиянием центров захвата на диффузию можно пренебречь~\cite{Heinola2010}. Тем не менее, значения коэффициента диффузии изотопов водорода сильно варьируются в литературе~\cite{remi_delaporte_mathurin_2024_13912922}, что значительно затрудняет проведение оценок накопления в материалах ОПЭ.

Облучение ОПЭ мощными тепловыми потоками неминуемо приведет к образованию температурных градиентов. Такая ситуация особо актуальна для конструкции элементов с активным водяным охлаждением, как моноблоки токамака ИТЭР. Градиент температур индуцирует поток термодиффузии $\mathbf{J}_T$ (эффект Соре):
\begin{equation}
    \mathbf{J}_{T} = -D\frac{\cm Q^*}{kT^2} \nabla T,
\end{equation}
направление которого определяется знаком теплоты переноса $Q^*$, \si{\electronvolt}.
\nomenclature[P, 11]{$Q^*$}{Теплота переноса, \si{\electronvolt}}

К сожалению, в настоящее время отсутствует общепринятое значение теплоты переноса водорода в вольфраме. Экспериментальные измерения показывают, что теплота переноса в металлах с положительной теплотой растворения отрицательна и линейно растет с температурой~\cite{Longhurst1985}. Оценки величины для вольфрама методом молекулярной динамики~\cite{Martinez2021,Dasgupta2023} также подтверждают отрицательное значение, но определенная функциональная зависимость пропорциональна квадрату температуры:
\begin{equation}
    \label{eq:ch1/heat_transport}
    Q^*(T)=-\num{0.0045} \, \kB T^2.
\end{equation}

\subsection{Центры захвата}

Дефекты в кристаллической решетке вольфрама, как вакансии, примеси, дислокации или пустоты, создают потенциальные энергетические ямы, более глубокие, чем междоузлия. Такие дефекты являются потенциальными <<ловушками>> для подвижных атомов водорода. Захваченные в <<ловушку>> атомы становятся неподвижными и могут покинуть ее, если тепловой энергии достаточно для преодоления потенциального барьера $E_\mathrm{dt}$ [\si{\electronvolt}]. Упрощенный процесс захвата можно представить в следующем виде~\cite{Drexler2020}:
\begin{equation*}
    \underset{\text{дефект}}{[\quad]} + \underset{\text{междоузлие}}{(\mathrm{H})} \mathop{\rightleftharpoons}^{\nu_\mathrm{t}}_{\nu_\mathrm{dt}}  \underset{\text{дефект}}{[\mathrm{H}]} +  \underset{\text{междоузлие}}{(\quad)},
\end{equation*}
где \( \nu_\mathrm{t} \) "--- константа скорости захвата в дефект, \si{\metre\cubed\per\second}; \( \nu_\mathrm{dt}\) "--- константа скорости выхода из дефекта, \si{\per\second}. Обычно полагается, что скорость захвата определяется диффузией: \( E_\mathrm{t} \approx E_\mathrm{D} \). Это простейшее представление оказывается весьма удобным при построении численных моделей удержания водорода в материале. Описание процессов многочастичного захвата~\cite{Johnson2010,Fernandez2015} и изотопного обмена также возможно путем рассмотрения отдельных энергетических уровней атомов в дефекте и введения соответствующих скоростей перехода между ними~\cite{Schmid2014}.

\nomenclature[P, 12]{\( \nu_\mathrm{t} \)}{Константа скорости захвата атомов в дефекты, \si{\metre\cubed\per\second}}
\nomenclature[P, 13]{\( \nu_\mathrm{dt} \)}{Константа скорость выхода атомов из дефектов, \si{\per\second}}
\nomenclature[P, 14]{\( E_\mathrm{t} \)}{Энергия активация захвата в дефект, \si{\electronvolt}}
\nomenclature[P, 15]{\( E_\mathrm{dt} \)}{Энергия активации выхода из дефекта, \si{\electronvolt}}

Дефекты в материале можно классифицировать на два типа. Первый тип, часто называемый собственными (естественными) дефектами, является свойственным материалу: примеси в сплавах, границы зерен поликристаллических металлов, элементарные дефекты малой концентрации, образованные в ходе производства. Второй тип дефектов, называемых внешними (индуцированными), возникает в результате внешних воздействий на материал, включая повреждения от бомбардировки частицами (ионами или нейтронами) или механического напряжения, и могут развиваться во времени и пространстве. Получение информации о параметрах центров захвата возможно экспериментально или путем моделирования. Атомистическое моделирование методом DFT позволяет рассчитать энергию связи атома с дефектом \( E_\mathrm{b} \), когда в рамках экспериментов наиболее надежно определяют барьер выхода из ловушек \( E_\mathrm{dt}=E_\mathrm{b} + E_\mathrm{t} \). Широко распространенным методом определения энергетического барьера выхода дефектов является термодесорбционная спектроскопия (ТДС). Дополнительную информацию о свойствах и пространственном распределении центров захвата позволяют получить ионно-пучковые методы анализа, например метод ядерных реакций (МЯР). 

\nomenclature[A, 8]{ТДС}{Термодесорбционная спектроскопия}
\nomenclature[A, 9]{МЯР}{Метод ядерных реакций}

Энергия связи атома водорода с дефектами зависит от их типа. Детальная информация об энергии связи приведена в исчерпывающих обзорах~\cite{Ogorodnikova2015,Li2020,Persianova2024}. Отметим наиболее распространенные типы атомистических дефектов в вольфраме. Наименьшей энергией связи (\SIrange{0.65}{1.25}{\electronvolt}) с атомами водорода характеризуются дислокации. Более сильная связь наблюдается между растворенными атомами растворенного водорода и вакансиями кристаллической решетки вольфрама с энергией связи \SIrange{1.05}{1.55}{\electronvolt}. Энергия связи увеличивается с ростом числа вакансий, агломерированных в вакансионный кластер, и спадает с числом атомов водорода, захваченных в них. Наибольшая энергия связи соответствует полостям в вольфраме с диапазоном значений от \num{2.05} до \SI{2.45}{\electronvolt}. Из приведенной краткой характеристики можно заметить, что литературные данные для различных дефектов пересекаются, что усложняет определение типа дефекта при экспериментальном анализе. Помимо этого, в ходе ионного и нейтронного облучения образуются каскады смещений, формируя сложные комплексы дефектов, а достижение определенной дозы накопленных изотопов водорода может инициировать образование микроскопических дефектов, как блистеры~\cite{Wang2001}. 

\subsection{Процессы на поверхности}\label{subsec:ch1/sec4/subsec3}

Существуют условия, когда процессы на поверхности могут во многом влиять на динамику изотопов водорода в материалах. Характерным примером являются эксперименты по накоплению изотопов водорода при экспозиции материалов в газовой среде, что является стандартным методом исследования водородного охрупчивания материалов\cite{Briant2002,Louthan2008}. В таких условиях энергии приходящих частиц обычно недостаточно для преодоления потенциального барьера имплантации, поэтому они с определенной вероятностью сорбируются на поверхности, образуя адсорбированные атомы. Кинетика процессов на поверхности в дальнейшем определяет эволюцию концентрации адсорбированных атомов. Схожим примером является процесс формирования молекулярного водорода в межзвездной среде. Известно, что при температурах, близких к нулю, характерных для плотных межзвездных облаков, обилие молекулярного водорода в основном объясняется эффективной рекомбинационной десорбцией атомов на поверхности пыли, протекающей более интенсивно по сравнению с другими процессами в окружающей газовой среде~\cite{Katz1999, Perets2005, Hama2013}. Противоположная ситуация наблюдается в капиллярных источниках атомарного водорода, основанных на конверсии молекул на горячей поверхности~\cite{Tschersich2000, Tschersich2008}. Было также показано, что учет процессов на поверхности может быть необходим при моделировании накопления в условиях ТЯУ~\cite{Guterl2019} и интерпретации некоторых результатов ТДС измерений после облучения образцов~\cite{Hodille2017, Matveev2018}. 

Для эндотермических металлов свободная поверхность является стоком атомов водорода из-за более низкого уровня потенциальной энергии по сравнению с растворенным состоянием (см. рисунок~\cref{fig:ch1/potential_diagram_all}). В процессе тепловой миграции внедренные атомы могут переходить в адсорбированное состояние из приповерхностной области при преодолении энергетического барьера \( E_\mathrm{bs}=E_\mathrm{s}-Q_\mathrm{s} \), где \( E_\mathrm{s} \) есть энергия активации растворения (в \si{\electronvolt}). Таким образом, адсорбированные атомы образуют отдельную фракцию, характеризующуюся поверхностной концентрацией \( \csurf \) (в \si{\per\meter\squared}). Величина энергии активации растворения может варьироваться в зависимости от наличия примесей в приповерхностной области, но в расчетах для <<чистой>> поверхности обычно полагается \( E_\mathrm{s} \approx E_\mathrm{D} + Q_\mathrm{s} \), что также определяет барьер перехода в адсорбированное состояние \(  E_\mathrm{bs} \approx E_\mathrm{D} \). Энергия связи адсорбированного атома с поверхностью определяется глубиной потенциальной ямы (теплотой) хемосорбции \( Q_\mathrm{c} \) (в \si{\electronvolt}). Наряду с выходом на поверхность возможен обратный процесс абсорбции, если энергии атома достаточно для преодоления барьера \( E_\mathrm{sb} = E_\mathrm{s} - Q_\mathrm{c} \). Переходы между адсорбированным и растворенным состояниями можно охарактеризовать аналогично процессам захвата и выхода из дефектов, введя константы скорости выхода на поверхность (\( \nubs \), \si{\meter\per\second}) и обратной абсорбции (\( \nusb \), \si{\per\second}), определяющие соответствующие потоки частиц.

Другой группой процессов, влияющих на эволюцию поверхностной концентрации, являются химическая адсорбция\footnote{Далее в работе термины <<химическая адсорбция>>, <<хемосорбция>> и <<адсорбция>> будут использованы как взаимозаменяемые, когда физическая адсорбция рассмотрена не будет.} и десорбция. На поверхности возможно протекание различных типов реакций с участием одного или нескольких адсорбированных атомов водорода (см. рисунок~\cref{fig:ch1/surface_processes}): диссоциативная адсорбция молекул водорода, рекомбинация двух адсорбированных атомов (рекомбинация Ленгмюра"--~Хиншельвуда), рекомбинация медленного атома/иона, приходящего на поверхность, и адсорбированного атома (рекомбинация Или"--~Ридила\footnote{В литературе данный тип десорбции обычно называется механизмом Или"--~Ридила, однако в некоторых источниках~\cite{Prins2018} отмечается, что в оригинальных работах Д. Или и Э. Ридила дается описание взаимодействия хемосорбированного и физиосорбированного атомов, когда для рассматриваемого процесса корректным названием будет <<механизм Ленгмюра"--~Ридила>>. В рамках данной работы будет использован термин <<механизм Или"--~Ридила>> из-за его большей распространенности.}), рекомбинация надтеплового (<<hot atom>>) атома с адсорбированным, распыление при облучении ионами, а также десорбция/адсорбция атомов. 

\begin{figure}[ht]
    \centerfloat{
        \includegraphics[scale=1]{surface_processes.png}
    }
    \caption{Схематическое представление некоторых процессов, происходящих на поверхности металла. Синие круги представляют падающие частицы (атомы или молекулы) из приповерхностной среды, зеленые круги соответствуют изначально адсорбированным атомам, серые "--- атомам материала поверхности}\label{fig:ch1/surface_processes}
\end{figure}

Плотность потока десорбированных частиц \( J_{\mathrm{des}} \) (в \si{\per\meter\squared\per\second}) можно описать в зависимости от порядка десорбции \( n \):
\begin{equation}
    \label{eq:ch1/des_flux}
    J_{\mathrm{des}} = n \nu_\mathrm{des} \csurf^n,
\end{equation}
где \( \nudes=\nu_{\mathrm{des},0} \exp \left( -E_\mathrm{des} / \kBT \right) \) "--- константа скорости десорбции, \(\text{м}^{2(n-1)}\cdot\text{с}^{-1}\); \( E_\mathrm{des} \) "--- энергия активации десорбции, \si{\electronvolt}. Упомянутые ранее механизмы десорбции соответствуют $n=1$ (десорбция атомов, механизм Или"--~Ридила, т.д.) и $n=2$ (механизм Ленгмюра"--~Хиншельвуда, реакция типа <<hot atom>>). Каналы десорбции молекул по схеме Ленгмюра"--~Хиншельвуда и атомов являются термически активируемыми. Также можно отметить, что в ряде работ~\cite{Baskes1980, Richards1988, Pisarev1997} рассматривались каналы десорбции с участием одного или двух атомов из приповерхностной области. 

Вероятность десорбции по каналу Ленгмюра"--~Хиншельвуда определяется теплотой хемосорбции и энергией активации хемосорбции \( E_\mathrm{c} \) (в \si{\electronvolt}). Согласно экспериментальным оценкам~\cite{Tamm1969,Tamm1971,Markelj2013} и моделированию методом DFT~\cite{Piazza2018,Ajmalghan2019,Ferro2023}, барьер десорбции молекул (\( E_\mathrm{des}=2(E_\mathrm{c}-Q_\mathrm{c}) \)) лежит в диапазоне \SIrange{0.25}{1.1}{\electronvolt} в зависимости от ориентации и состояния поверхности, а также в среднем спадает с ростом концентрации адсорбированного водорода. Детальный анализ влияния конфигурации адсорбированных атомов на барьер десорбции указывает на его снижение при кластеризации атомов на поверхности~\cite{Degtyarenko2024,Stihl2021}.  Обратный механизм протекает при диссоциации молекул у поверхности при преодолении барьера \( E_\mathrm{diss}=2E_\mathrm{c} \) и последующей адсорбцией двух атомов. В случае чистой поверхности адсорбция частиц происходит безактивационно (\( E_\mathrm{c} \approx 0 \))~\cite{Piazza2018,Ajmalghan2019,Ferro2023}. Десорбция атомов требует преодоления гораздо большего барьера (\( E_\mathrm{des}=E_\mathrm{d}-Q_\mathrm{c} \)), что возможно при высоких температурах поверхности ввиду относительно большой разницы между энергетическими уровнями (\(E_\mathrm{d}=\SI{2.26}{\electronvolt}\)) атома водорода в свободном и связанном в молекуле состоянии.

При локальном равновесии вблизи поверхности и пределе малой доли водорода в металле концентрации на поверхности и в приповерхностной области связаны следующим соотношением~\cite{Pick1985}:
\begin{equation}
    \label{eq:ch1/bs_equilibrium}
    \csurf = \cm \frac{\nubs}{\nusb}.
\end{equation}
Подстановка данного выражения в уравнение~\eqref{eq:ch1/des_flux} приводит к:
\begin{equation}
    J_{\mathrm{des}} = n \nu_\mathrm{des} \left( \frac{\nubs}{\nusb} \right)^n \cm^n=K_\mathrm{r} \cm^n,
\end{equation}
где \( K_\mathrm{r} \) "--- коэффициент рекомбинации на поверхности, \si{\meter^{3n-2}}. Применение коэффициента рекомбинации оправдано, когда процессы в объеме протекают гораздо медленнее процессов на поверхности. Данный подход позволяет упростить задачу транспорта изотопов водорода в материалах, что удобно при проведении численного моделирования. Наиболее распространенными в литературе коэффициентами рекомбинации водорода на поверхности вольфрама (в \si{\meter^4\per\second}) являются аналитический коэффициент рекомбинации Пика-Сонненберга~\cite{Pick1985}:
\begin{equation}
    \label{eq:ch1/Kr_PS}
    K_\mathrm{r}(T) = \frac{\num{3.0e-25}}{\sqrt{T}} \exp \left( \frac{Q_\mathrm{s}-E_\mathrm{c}}{\kBT} \right),
\end{equation}
и эмпирический коэффициент Андерла~\cite{Anderl1992}
\begin{equation}
    K_\mathrm{r}(T) = \num{3.2e-15} \exp \left( -\frac{\SI{1.16}{[\electronvolt]}}{\kBT} \right).
\end{equation}
Приведенные выражения имеют совершенно разную функциональную зависимость от температуры, причем показатель экспоненты в выражении~\cref{eq:ch1/Kr_PS} становится отрицательным исключительно при наличии большого барьера хемосорбции на поверхности. В работе~\cite{Ogorodnikova2019} продемонстрировано, что применение коэффициента рекомбинации Андерла приводит к существенному завышению проникающего потока, что вызывает несоответствие с экспериментальными данными.

Помимо применения коэффициентов рекомбинации, распространенной аппроксимацией процессов на поверхности является приближение бесконечно быстрой рекомбинации. Качественное представление можно получить, рассмотрев точечный источник имплантированных атомов с плотностью потока \( \Gamma \) (в \si{\per\meter\squared\per\second}). При достижении равновесия поток атомов, диффундирующих вглубь материала становится мал, а поток десорбированных частиц уравновешивается потоком имплантируемых. Учитывая равенство~\cref{eq:ch1/bs_equilibrium} при локальном равновесии и полагая, что с поверхности десорбируются двуатомные молекулы, можно получить выражение для максимальной концентрации атомов растворенного водорода вблизи поверхности: 
\begin{equation}
    \cm = \sqrt{\frac{\Gamma}{K_\mathrm{r}}},
\end{equation}
которое в пределе бесконечно большой скорости рекомбинации упрощается до вида: \( \cm=0 \).

Важной величиной является растворимость водорода в вольфраме. Выражение для растворимости можно получить, рассмотрев поверхность, находящуюся в равновесии с газом при давлении \( p \) (в \si{\pascal}). Поток частиц \( J_\mathrm{ads} \) (в \si{\per\meter\squared\per\second}), сорбирующихся на поверхность из газовой фазы, имеет следующее представление: 
\begin{equation}
    J_\mathrm{ads} = n \nu_{\mathrm{ads}} p,
\end{equation}
где \( \nu_{\mathrm{ads}} \) "--- константа скорости адсорбции молекул из газовой фазы, \si{\per\meter\squared\per\second\per\pascal}. При локальном равновесии потоки адсорбирующихся и десорбирующихся частиц равны. Учитывая выражения~\eqref{eq:ch1/des_flux} и \eqref{eq:ch1/bs_equilibrium}, можно получить:
\begin{equation}
    \cm = \sqrt[n]{\frac{\nu_\mathrm{ads}}{\nu_\mathrm{des}}}\sqrt[n]{p}.
\end{equation}
Для двуатомных молекул водорода выражение упрощается до вида: \( \cm = K_\mathrm{s} \sqrt{p} \), описывающее закон растворения Сивертса с константой \( K_\mathrm{s}=K_{\mathrm{s},0} \exp \left( -Q_\mathrm{s}/\kBT \right) \). Наиболее распространенное значение растворимости (в \( \si{\per\meter\cubed}\cdot\si{\pascal}^{-0.5} \)) было измерено Фраунфельдером~\cite{frauenfelder1969solution}:
\begin{equation}
    K_\mathrm{s}(T) = \num{8.88e23} \exp \left( -\frac{\SI{1.04}{[\electronvolt]}}{\kBT} \right).
\end{equation}

\nomenclature[P, 16]{\( E_\mathrm{bs} \)}{Энергетический барьер перехода из растворенного состояния в адсорбированное, \si{\electronvolt}}
\nomenclature[P, 17]{\( E_\mathrm{sb} \)}{Энергетический барьер перехода из адсорбированного состояния в растворенное, \si{\electronvolt}}
\nomenclature[P, 18]{\( E_\mathrm{s} \)}{Энергия активации растворения, \si{\electronvolt}}
\nomenclature[P, 19]{\( Q_\mathrm{C} \)}{Теплота хемосорбции, \si{\electronvolt}}
\nomenclature[P, 20]{\( \nubs \)}{Константа скорости перехода из растворенного состояния в адсорбированное, \si{\meter\per\second}}
\nomenclature[P, 21]{\( \nusb \)}{Константа скорости перехода из адсорбированного состояния в растворенное, \si{\per\second}}
\nomenclature[P, 22]{\( \csurf \)}{Поверхностная концентрация адсорбированных атомов, \si{\per\meter\squared}}
\nomenclature[P, 23]{\( J_\mathrm{des} \)}{Поток десорбированных атомов, \si{\per\meter\squared\per\second}}
\nomenclature[P, 24]{\( J_\mathrm{ads} \)}{Поток адсорбирующихся (хемосорбирующихся) атомов, \si{\per\meter\squared\per\second}}
\nomenclature[P, 25]{\( E_\mathrm{des} \)}{Энергетический барьер десорбции, \si{\electronvolt}}
\nomenclature[P, 26]{\( K_\mathrm{r} \)}{Коэффициент рекомбинации, \si{\meter^4\per\second}}
\nomenclature[P, 27]{\( K_\mathrm{s} \)}{Константа растворимости, \(\si{\per\meter\cubed\pascal}^{-0.5}\)}
\nomenclature[P, 28]{\( p\)}{Давление, \si{\pascal}}
\nomenclature[P, 29]{\( E_\mathrm{diss} \)}{Энергия активации диссоциации молекулы с последующей хемосорбцией двух атомов, \si{\electronvolt}}
\nomenclature[P, 30]{\( E_\mathrm{d} \)}{Энергия диссоциации молекулы водорода, \si{\electronvolt}}

\section{Накопление изотопов водорода в вольфраме}\label{sec:ch1/sec5}

Краткий обзор основных механизмов взаимодействия указывает на комплексность процессов, определяющих захват изотопов водорода в вольфраме. Динамика накопления изотопов водорода в вольфраме была и является предметом интенсивных научных исследований. Условия облучения и подготовка образцов определяют долю захваченных атомов. На рисунке~\cref{fig:ch1/retention_fluence} приведены результаты ряда экспериментов, проведенных при ионном и плазменном облучении. Наблюдается широкий разброс в интегральных значениях при различных условиях эксперимента, однако скорость накопления в них хорошо согласуется с корневой зависимостью от дозы облучения, что соответствует диффузионному механизму.  

\begin{figure}[ht]
    \centerfloat{
        \includegraphics[scale=0.35]{retention_fluence.png}
    }
    \caption{Зависимость интегрального содержания дейтерия в поликристаллическом вольфраме от дозы облучения: а) ионным пучком при комнатной температуре; б) ионным пучком и плазмой при температуре в диапазоне \SIrange{380}{643}{\kelvin}~\cite{HarutunyanThesis}}\label{fig:ch1/retention_fluence}
\end{figure}

Содержание водорода в вольфраме во многом определяется динамикой развития центров захвата. Облучение поверхности до больших доз может приводить к образованию слоя перенасыщения в области внедрения (\( \sim \SI{10}{\nano\meter} \)), когда концентрация атомов превышает предел растворимости. Предполагается, что развитие процесса происходит за счет индуцирования напряжений в приповерхностной области, облегчающих создание новых центров захвата для внедряемых атомов~\cite{Nishijima2023}. Дальнейшее развитие поверхности может приводить к появлению блистеров. Цикличные процессы образования и разрушения блистеров влияют на динамику удержания в приповерхностной области, а также препятствуют распространению атомов вглубь материала~\cite{Bauer2017}.  

Несмотря на возможное достижение высокой концентрации (\( \approx \SI{10}{\text{ат.}\percent} \)) захваченных атомов, глобальное удержание топлива не должно сильно зависеть от этих приповерхностных механизмов ввиду малой толщины затрагиваемой области. Во время работы реактора образование дефектов будет происходить по всему объему ОПЭ в результате непрерывной бомбардировки нейтронами высокой энергии. В действующих токамаках нет возможности для исследования эффекта термоядерных нейтронов на накопление изотопов водорода в ОПЭ. Распространенным подходом для имитации повреждений, создаваемых при нейтронном облучении в приповерхностной области материалов, является использование ионов тяжелых элементов с энергией порядка нескольких мегаэлектронвольт. Результаты некоторых экспериментов по накоплению дейтерия после предварительного облучения МэВ-ными ионами приведены на рисунке~\cref{fig:ch1/retention_dpa}. Индуцированное число смещений на атом в ИТЭР оценивается на уровне \num{0.6} и \num{1.0} для дивертора и первой стенки, соответственно. Экстраполяция экспериментальных данных на указанный диапазон показывает, что концентрация захваченных изотопов водорода в вольфраме может достигать \( \sim \SI{1}{\text{ат.}\percent} \). 

\begin{figure}[ht]
    \centerfloat{
        \hfill
        \subcaptionbox[List-of-Figures entry]{\label{fig:ch1/retention_dpa}}{%
            \includegraphics[width=0.49\linewidth]{retention_dpa.png}}
        \hfill
        \subcaptionbox{\label{fig:ch1/retention_HeTemperature}}{%
            \includegraphics[width=0.49\linewidth]{retention_HeTemperature.png}}
    }
    \caption{Зависимость содержания захваченного дейтерия в вольфраме от а) числа смещений на атом, индуцированных предварительным облучением МэВ-ными ионами~\cite{Roth2011}, и б) температуры~\cite{Rieth2019}}
\end{figure}

Процесс накопления изотопов водорода также сильно зависит от температуры, при которой происходит облучение. Как было показано ранее, процессы диффузии, захвата в дефекты и десорбции протекают интенсивнее при более высокой температуре. В условиях сильного нагрева материала возможен отжиг имеющихся дефектов, что будет снижать эффективность удержания изотопов водорода~\cite{Dark2024}. Совокупность экспериментальных результатов на рисунке~\cref{fig:ch1/retention_HeTemperature} демонстрирует снижение скорости накопления дейтерия в вольфраме в области температур более \SI{600}{\kelvin}. Примечательно, что интегральное удержание водорода также может быть заметно снижено за счет добавления примеси гелия в поток ионов, что также подавляет образование блистеров на поверхности вольфрама~\cite{Baldwin2011}. 

Анализ накопления изотопов водорода при стационарном облучении вольфрама также широко исследовался методом численного моделирования. Распространенным подходом является численный расчет динамики транспорта изотопов водорода на основе реакционно-диффузионной модели при использовании фиксированных параметров потоков тепла и частиц, ожидаемых в условиях отдельной установки. Такой подход применялся для исследования длительного (временные масштабы достигают десятков тысяч импульсов в ИТЭР) накопления как в комплексной дву-/трехмерной геометрии отдельных диверторных моноблоков ИТЭР~\cite{Delaporte-Mathurin2019,Hodille2021_2,Delaporte-Mathurin2021}, так и интегрального накопления в диверторе~\cite{Delaporte-Mathurin2020, Delaporte-Mathurin2021_2}. Полученные результаты демонстрируют, что интегральное накопление трития в диверторной области может достигать около \SI{10}{\gram}, что существенно ниже регламентированного предела в \SI{1}{\kilogram}. Также стоит отметить, что в работе~\cite{Hodille2021_2} было продемонстрировано отсутствие значимой разницы между скоростью накопления при моделировании полных плазменных импульсов, учитывающем фазы нарастания и спада нагрузок, и непрерывного облучения, что позволяет упростить расчеты и не рассматривать медленные переходные процессы такого типа. Более точные результаты можно получить путем согласованного моделирования процессов в периферийной плазме и ОПЭ~\cite{Smirnov2020,Lasa2024,Smirnov2024}. Однако расчеты такого рода характеризуются большой вычислительной сложностью и требуют соответствующих вычислительных ресурсов, что не позволяет проводить анализ на больших временных масштабах, соответствущих длительным (\SI{100}{\second} и более) плазменным разрядам.

В свою очередь, влияние импульсных плазменных нагрузок на накопление изотопов водорода изучено менее детально. Основной интерес представляют ELM"=события, развитие которых ожидается при переходе в режим с повышенным удержанием энергии в плазме токамака. Ранее отмечалось, что ELM"=события сопровождаются периодическим выбросом горячей плазмы на ОПЭ, создающим импульсные потоки тепла и частиц в дополнение к стационарным. Полностью воспроизвести параметры облучения, ожидаемые в установках масштаба ИТЭР, на данный момент не представляется возможным, поэтому применяются различные подходы по имитации воздействия импульсных нагрузок.

Эксперименты по анализу влияния импульсных тепловых нагрузок во время стационарного плазменного облучения вольфрама ионами дейтерия проводились на установке PSI-2 при плотности потока ионов \SI{6e21}{\per\meter\squared\per\second} с энергией \SI{60}{\electronvolt}~\cite{Huber2016_1, Huber2016_2}. Нагрев осуществлялся при помощи лазерных импульсов с прямоугольным временным профилем, длительностью \SI{1}{\milli\second} и частотой повторения \SI{0.5}{\hertz} с плотностью мощности в диапазоне \SIrange{0.19}{0.86}{\giga\watt\per\meter\squared}. При совместном плазменном и лазерном воздействии накопление дейтерия в вольфраме увеличивалось в несколько раз по сравнению со случаем без лазерного воздействия. В работах изначально полагалось, что увеличение доли накопленного дейтерия происходит из-за эффектов термических ударов, образования блистеров и трещин, а также большей подвижности внедренных атомов дейтерия из-за дополнительного нагрева материала. Последующее сравнение результатов измерений методами ТДС и МЯР показало, что эффекты, связанные с влиянием модификации поверхности, вносят второстепенный вклад, так как повышенный захват дейтерия происходит в объеме материала. Противоположная ситуация наблюдалась в моделировании влияния ELM"=событий на накопления дейтерия путем решения задачи транспорта~\cite{Hu2015}. В работе не учитывалось повышение потока частиц во время ELM"=событий, а их влияние было сведено к импульсным изменениям температуры поверхности, которые приводили к снижению скорости накопления дейтерия.

Существенное накопления наблюдалось при облучении вольфрамовых образцов импульсными потоками дейтериевой плазмы в линейном плазменном ускорителе КСПУ-Т~\cite{Ogorodnikova}. Плотность энергии в ходе экспериментов варьировалась в диапазоне \SIrange{0.3}{3.7}{\mega\joule\per\meter\squared} при длительности импульса равной \SI{1}{\milli\second}. Плотность потока частиц была оценена на уровне \SI{7.5e26}{\per\meter\squared\per\second} при их энергии менее \SI{10}{\electronvolt}~\cite{Poskakalov2020}. Количество импульсов достигало 30 с интервалом 15 минут между ними, что обеспечивало охлаждение образцов до комнатной температуры перед следующим воздействием. Сравнение результатов экспериментов в режиме с плавлением и без (красные пунктирная и сплошная линии) с данными, полученными при стационарном плазменном облучении (черная линия), приведено на рисунке~\cref{fig:ch1/retention_QSPA}. 
\begin{figure}[ht]
    \centerfloat{
        \includegraphics[scale=0.75]{retention_QSPA.png}
    }
    \caption{Зависимость дозы накопленного дейтерия в вольфраме после импульсного плазменного воздействия в режимах без плавления (пунктирная красная линия) и с плавлением (сплошная красная линия) от количества импульсов в сравнении с соответствующей зависимостью после воздействия плазмы в зависимости от дозы облучения~\cite{Ogorodnikova}}\label{fig:ch1/retention_QSPA}
\end{figure}
Значительный захват наблюдается при одном импульсе в обоих режимах. Однако при увеличении числа импульсов в режиме без плавления накопление дейтерия уменьшается. Это, как предполагается в работе, связано с образованием трещин, которые увеличивают площадь свободной поверхности и облегчают десорбцию. В режиме с плавлением, напротив, наблюдается рост накопления дейтерия. Это объясняется более интенсивной диффузией из области расплава в объем вольфрама. Проникновение на большую глубину (\( \sim \SI{10}{\micro\meter} \)) за один импульс было численно продемонстрировано в аналогичной работе~\cite{Poskakalov2020}, в которой также был представлен эмпирический факт о росте доли захваченного дейтерия с увеличением плотности энергии плазменного импульса с \num{0.4} до \SI{3.7}{\mega\joule\per\metre\squared}. Однако абсолютно противоположная ситуация наблюдалась в более ранних экспериментах на установке MCPG~\cite{Nishijima2011}. Вольфрамовые образцы облучались десятью плазменными импульсами длительностью \( \approx\SI{0.5}{\milli\second} \) при параметрах потока ионов, близких к условиям в КСПУ-Т: плотность потока порядка \SI{e26}{\per\meter\square\per\second} при энергии порядка \SI{10}{\electronvolt}. Исходя из анализа полученных данных, доза захваченного дейтерия уменьшилась более чем в четыре раза при увеличении плотности энергии импульса с \num{0.5} до \SI{0.7}{\mega\joule\per\meter\squared}.

Анализ захвата изотопов водорода во время переходных процессов в токамаках также проводят путем решения задачи транспорта в ОПЭ, используя параметры облучения, полученные с диагностических зондов, или при совместном моделировании процессов в пристеночной плазме и в объеме. Расчеты с использованием детальной модели транспорта водорода и экспериментальных параметров облучения в JET не ставили целью анализ динамики интегрального накопления, но показали, что повышение температуры во время ELM"=событий приводит к агломерации и повышению концентрации сложных дефектов с более высокой энергией связи в приповерхностном слое~\cite{Heinola2019}. Численный анализ динамики удержания с использованием экспериментальных данных токамака EAST, по всей видимости, не учитывающий временную эволюцию температуры, продемонстрировал увеличение интегрального накопления примерно на \SI{20}{\percent} во время ELM"=событий из-за большей вероятности внедрения быстрых частиц и образования ион-индуцированных дефектов~\cite{Sang2014}. Моделирование накопления в условиях, имитирующих ELM"=события в токамаке JET, продемонстрировало незначительное снижении скорости накопления дейтерия и флуктуации температуры поверхности с амплитудой \( \approx \SI{100}{\kelvin} \)~\cite{Schmid2016}. В работе также показано, что коэффициент рециклинга быстро достигает и колеблется вблизи единицы. Такая же динамика изменения коэффициента рециклинга наблюдается и в более поздних работах по исследованию влияния развития ELM"=неустойчивости в магнитной геометрии токамака DIII-D на основе согласованного моделирования процессов в периферийной плазме и ОПЭ~\cite{Smirnov2020,Lasa2024,Smirnov2024}. Отмечается, что амплитуда колебаний коэффициента рециклинга определяется областью прихода нагрузок во время ELM-событий в ходе разряда: наибольшии флуктуации наблюдались в <<холодных>> областях, как первая стенка установки. Динамическое удержание дейтерия в ОПЭ может играть решающую роль в балансе частиц плазмы на периферии во время ELM"=цикла и, следовательно, влиять на время восстановление пьедестала между ELM-событиями.

Таким образом, небольшой объем имеющихся данных не позволяет однозначно охарактеризовать влияние импульсных плазменных нагрузок на накопление изотопов водорода в вольфраме. Параметры облучения в лабораторных установках не могут в полной мере воспроизвести условия, характерные для токамаков. Импульсная природа установок не позволяет напрямую имитировать дополнительные нагрузки во время стационарного облучения для проведения сравнительного анализа, а низкая частота повторения импульсов сводит задачу к исследованию влияния отдельных событий на сравнительно малых временных масштабах. Согласованное моделирование процессов в периферийной области токамаков позволяет получить более детальную информацию, но также на малых временных масштабах. Стоит также подчеркнуть, что результаты численных расчетов по оценке влияния импульсных нагрузок были получены для действующих токамаков, параметры которых существенно отличаются от ожидаемых в установках масштаба ИТЭР, что определяет актуальность проведения дальнейших исследований.

\section{Лазерно-индуцированная десорбция}\label{sec:ch1/sec6}

Развитие методов анализа содержания изотопов водорода является задачей, непосредственно связанной с вопросом их накопления в элементах вакуумной камеры установки. Глобальные измерения содержания изотопов водорода осуществляются методом газового баланса на основе потоков введенных в камеру и откаченных частиц. Различные методы \textit{post-mortem} анализа предоставят информацию о локальном количестве удержанного водорода в извлеченных элементах. Подобные методы анализа предоставляют только данные об интегральном накоплении в ходе кампании и обычно не могут быть использованы для характеризации отдельных плазменных разрядов. Следует отметить, что извлекаемые из вакуумной камеры образцы, как правило, подвергаются воздействию атмосферы во время транспортировки и хранения, что может усложнять интерпретацию результатов измерений.

Осуществление \textit{in situ} анализа возможно при помощи методов, основанных на взаимодействии лазерного излучения с поверхностью внутрикамерных элементов. Применение лазерных методов позволяет проводить локальное пробирование участков поверхности, когда получение информации о пространственных параметрах может быть получено путем сканирования различных областей первой стенки. Среди методов, разрабатываемых для установок с магнитным удержанием плазмы, можно выделить лазерно-индуцированную десорбцию (ЛИД), лазерно-индуцированную абляцию (ЛИА) и лазерно-искровую эмиссионную спектроскопию (ЛИЭС)~\cite{Philipps2013,Mukhin2016}. Схематическое представление каждого из методов приведено на рисунке~\cref{fig:ch1/laser_methods}.
\nomenclature[A, 10]{ЛИА}{Лазерно-индуцированная абляция}
\nomenclature[A, 11]{ЛИЭС}{Лазерно-искровая эмиссионная спектроскопия}
\nomenclature[A, 12]{КМС}{Квадрупольный масс-спектрометр}

\begin{figure}[ht]
    \centerfloat{
        \includegraphics[scale=1]{laser_methods.png}
    }
    \caption{Схематическое представление лазерно-ассистированных методов анализа содержания изотопов водорода в ОПЭ}\label{fig:ch1/laser_methods}
\end{figure}

Основные процессы, лежащие в основе различных методов, можно условно дифференцировать в зависимости от плотности мощности лазерного излучения, но важно заметить, что границы применимости каждого из подходов зависят от материала и состояния поверхности. В ЛИЭС обычно используется интенсивное (\( >\SI{1}{\giga\watt\per\meter\squared} \)) наносекундное лазерное облучение, которое ведет к образованию плазменного факела. Определение состава плазмы происходит на основе спектроскопии эмитированного излучения. Данная диагностика позволяет проводить анализ по глубине и была апробирована на множестве токамаков~\cite{Xiao2015,Semerok2016,Imran2023,Favre2024}, однако точность интерпретации результатов измерений сильно зависит от параметров образующейся плазмы~\cite{Marenkov2021}. 

ЛИА проходит в промежуточном диапазоне плотности мощности лазерного излучения (\SIrange{0.5}{1.0}{\giga\watt\per\meter\squared}) с наносекундной длительностью и сопровождается только абляцией материала исследуемой поверхности и десорбцией захваченного газа за счет нагрева. Аблированные атомы затем детектируются напрямую квадрупольным масс-спектрометром (ЛИА-КМС) или за счет оптической спектроскопии излучения при ионизации в приповерхностной плазме (ЛИАС). Метод также позволяет проводить анализ содержания по глубине за счет послойной абляции поверхности. ЛИА была успешно применена на таких токамаках, как TEXTOR~\cite{Gierse2016}, EAST~\cite{Hu2018} и Глобус-М2~\cite{Medvedev2024}, демонстрируя перспективность подхода для будущих установок. 

По сравнению с двумя предыдущими диагностиками, ЛИД является неразрушающим и более простым (в плане постобработки) методом, базирующемся на нагреве поверхности. Импульсный нагрев инициирует десорбцию захваченных атомов изотопов водорода, которые затем анализируются при помощи КМС (ЛИД-КМС) или оптической спектроскопии (ЛИДС). Применимость ЛИД была продемонстрирована на ряде токамаков~\cite{Schweer2009, Medvedev2024}, в том числе и на JET~\cite{Zlobinski2024}. Учитывая перспективность метода, соответствующие диагностические комплексы разрабатываются для токамака ИТЭР и предложены для токамака ТРТ~\cite{Razdobarin2022}. 

Эффективность ЛИД (доля десорбированных атомов по отношению к их начальному количеству) дейтерия из соосажденных бериллиевых пленок определялась при длительности лазерного нагрева в диапазоне \SIrange{1}{10}{\milli\second}~\cite{Zlobinski2019, Zlobinski2020}. Результаты экспериментов, приведенные на рисунке~\cref{fig:ch1/LID_efficiency}, демонстрируют практически полный выход захваченного дейтерия за один импульс при достижении температуры плавления бериллия. 
\begin{figure}[ht]
    \centerfloat{
        \hfill
        \subcaptionbox[List-of-Figures entry]{\label{fig:ch1/LID_efficiency_thin}}{%
            \includegraphics[width=0.49\linewidth]{LID_efficiency_thin.png}}
        \hfill
        \subcaptionbox{\label{fig:ch1/LID_efficiency_thick}}{%
            \includegraphics[width=0.49\linewidth]{LID_efficiency_thick.png}}
    }
    \caption{Зависимость числа десорбированных атомов дейтерия из соосажденных пленок бериллия от плотности поглощенной энергии лазерного импульса различной длительности: а) толщина пленки \SI{1}{\micro\meter}, содержание дейтерия \SIrange{25}{30}{\text{ат.}\percent}~\cite{Zlobinski2019}; б) толщина пленки \SI{10}{\micro\meter}, содержание дейтерия \SI{1}{\text{ат.}\percent}~\cite{Zlobinski2020}}\label{fig:ch1/LID_efficiency}
\end{figure}
Применение более длительных импульсов (\SI{1}{\second}) при анализе эффективности десорбции дейтерия из соосажденных пленок вольфрама (толщина \SI{1.2}{\micro\meter}, содержание дейтерия \( \approx \SI{1.7}{\text{ат.}\percent} \)) показало эффективность десорбции на уровне \SI{97}{\percent} за один импульс~\cite{Yu2019}.

Высокая эффективность дегазации толстых пленок, образование которых возможно при распылении материалов ОПЭ во время длительных плазменных разрядов в ТЯУ, говорит о необходимости использования более долгих лазерных импульсов для определения интегрального накопления в них. Рассматривается возможность использования наносекундных лазерных импульсов для ЛИД~\cite{Medvedev2024, Gasparyan2021, Efimov2024}. При таком воздействии глубина прогрева материала составляет \SIrange{50}{100}{\nano\meter}, что позволяет анализировать тонкий поверхностный слой. В то же время, возможной становится реализация режима ЛИА. Объединение нескольких методов диагностики в рамках одного комплекса расширяет возможности анализа, а также облегчает проектирование и распределение компонентов крупных установок типа токамак.

Открытым остается вопрос о влиянии параметров материала и центров захвата изотопов водорода в нем на эффективность ЛИД. Известно, что структура соосажденных слоев отличается от идеальной кристаллической. Помимо этого, в условиях, характерных для ТЯУ, будет происходить непрерывная эволюция параметров дефектов, определяющих удержание изотопов водорода. Соответственно, исследование закономерностей выхода изотопов водорода из материалов ОПЭ актуально для определения эффективных режимов и оценки возможных источников погрешности ЛИД. 

\section{Выводы к главе 1}

Совокупность физических свойств вольфрама определяет перспективность его использование для элементов облицовки ТЯУ. В настоящее время он выбран в качестве основного материала ОПЭ токамака ИТЭР. Оценки накопления радиоактивного трития в ИТЭР не учитывают в полной мере весь спектр процессов, протекание которых ожидается при достижении проектных параметров установки. 

Существенным пробелом является отсутствие информации о влиянии импульсных плазменных нагрузок во время быстрых переходных процессов, как ELM"=события, сопровождаемых переходом в режим с улучшенным удержанием энергии в плазме токамаков. Проблема усугубляется технической сложностью реализации импульсных плазменных нагрузок, релевантных условиям в масштабных установках, как ИТЭР. 

На большинстве современных токамаках идет активная разработка и апробация методов дистанционного контроля содержания изотопов водорода. Среди них лазерно-индуцированная десорбция является перспективным методом, позволяющим проводить анализ содержания без нанесения существенного ущерба поверхности. Исследование закономерностей выхода изотопов водорода при лазерном нагреве актуально для определения наиболее эффективных режимов диагностики и выявления возможных источников погрешности.

\FloatBarrier
