\chapter{Захват дейтерия в вольфраме под действием импульсно-периодических плазменных нагрузок}\label{ch:ch3}

\section{Валидация}\label{sec:ch3/sec1}
\subsection{Детали эксперимента}\label{sec:ch3/sec1/subsec1}
\subsection{Расчетная модель}\label{sec:ch3/sec1/subsec2}
\subsection{Сравнение результатов моделирования и эксперимента}\label{sec:ch3/sec1/subsec3}

%\section{\fixme{Постановка задачи}}\label{sec:ch3/sec1}
%\subsection{Геометрическая модель}\label{sec:ch3/sec1/subsec1}
%\subsection{Стационарные плазменные нагрузки}\label{sec:ch3/sec1/subsec2}
%\subsection{Импульсно-периодические плазменные нагрузки}\label{sec:ch3/sec1/subsec3}

\section{Моделирование накопления дейтерия в вольфраме под действием импульсно-периодических плазменных нагрузок}\label{sec:ch3/sec2}
\subsection{Постановка задачи}\label{sec:ch3/sec2/subsec1}
\subsection{Эволюция температуры}\label{sec:ch3/sec2/subsec2}
\subsection{\fixme{Коэффициент рециклинга}}\label{sec:ch3/sec2/subsec3}
\subsection{Влияние параметров плазменных нагрузок}\label{sec:ch3/sec2/subsec4}
\subsection{Влияние параметров центров захвата и скорости рекомбинации на поверхности}\label{sec:ch3/sec2/subsec5}


\section{Аналитический анализ}\label{sec:ch3/sec3}
\subsection{Квазистационарное приближение}\label{sec:ch3/sec3/subsec1}
\subsection{Распределение концентрации дейтерия при насыщении}\label{sec:ch3/sec3/subsec2}
\subsection{Сравнение аналитического решения с результатами численного расчета}\label{sec:ch3/sec3/subsec3}

\section{Выводы к Главе~\ref{ch:ch3}}

\clearpage
