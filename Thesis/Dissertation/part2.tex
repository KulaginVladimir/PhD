\chapter{Методика анализа динамики транспорта дейтерия в вольфраме}\label{ch:ch2}

Данная глава посвящена описанию математических моделей и программного пакета FESTIM, использованных в рамках данной диссертации для анализа транспорта изотопов водорода в материалах. Описана реализация модели, учитывающей поверхностные процессы, в коде FESTIM, а также представлены результаты ее верификации и валидации. 

\section{Модель транспорта изотопов водорода в материалах}\label{sec:ch2/sec1}
\subsection{Объемные процессы}\label{sec:ch2/sec1/subsec1}

\subsubsection{Транспорт водорода}\label{sec:ch2/sec1/subsec1/subsubsec1}

\begin{subequations}
    \label{eq:ch2/hydrogen_transport}
    \begin{align}
        \frac{\partial \cm}{\partial t} & = \nabla \cdot \left( D \nabla \cm \right) - \sum \limits_i \frac{\partial \cti}{\partial t} + \sum \limits_j S_j, \label{eq:ch2/mobile_conc} \\
        \frac{\partial \cti}{\partial t} & = k_i \cm (n_i - \cti) - p_i \cti, \label{eq:ch2/trapped_conc} 
    \end{align}     
\end{subequations}
  
Уравнения \cref{eq:ch2/mobile_conc,eq:ch2/trapped_conc} в системе \eqref{eq:ch2/hydrogen_transport}


\subsubsection{Перенос тепла}\label{sec:ch2/sec1/subsec1/subsubsec2}

\begin{equation}
    C \rho \frac{\partial T}{ \partial t} = \nabla \cdot \left( \kappa \nabla T \right) + \sum \limits_i Q_i
\end{equation}


\subsection{Поверхностные процессы}\label{sec:ch2/sec1/subsec2}

\section{Программный пакет FESTIM}\label{sec:ch2/sec2}

\section{Реализация нульмерной модели, учитывающей поверхностные процессы, в программном пакете FESTIM}\label{sec:ch2/sec3}
Результаты анализа корректности имплементации модели распространяются свободно~\cite{vladimir_kulagin_2025_14738004} и включены в онлайн-книгу с результатами верификации и валидации кода FESTIM~\cite{FESTIM_VV}.

\subsection{Верификация модели}\label{sec:ch2/sec3/subsec1}
\subsection{Валидация модели}\label{sec:ch2/sec3/subsec2}
\subsubsection{Эксперимент по абсорбция протия в титане}\label{sec:ch2/sec3/subsec2/subsubsec1}
\subsubsection{Эксперимент по адсорбция дейтерия на поверхности оксидированного вольфрама}\label{sec:ch2/sec3/subsec2/subsubsec2}
\subsubsection{Эксперимент по облучению вольфрама низкоэнергетичными атомами дейтерия}\label{sec:ch2/sec2/subsec3/subsubsec3}
\subsubsection{Эксперимент по облучению стали EUROFER ионами дейтерия}\label{sec:ch2/sec2/subsec/subsubsec4}

\section{\fixme{Выводы к Главе~\ref{ch:ch2}}}

\FloatBarrier
