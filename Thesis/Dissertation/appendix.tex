\chapter{Аналитические выражения для температурной зависимости теплофизических свойств материалов}\label{app:A}

Температурная зависимость теплофизических свойств материалов задавалась на основе полиномиальных выражений. Далее плотность материала \( \rho \) представлена в \si{\kilogram\per\meter\cubed}, удельная теплоемкость \( C_p \) "--- в \si{\joule\per\kilogram\per\kelvin}, теплопроводность \( \kappa \) "--- в \si{\watt\per\meter\per\kelvin}. Изменением плотности материала за счет расширения при нагреве во всех случаях пренебрегалось. Для вольфрама использовались функциональные зависимости, рекомендованные Толиасом~\cite{Tolias2017} на основе критического анализа литературных данных:
\begin{subequations}
    \label{eq:app/W_props}
    \begin{align}
        \rho  =             & \num{19250},                                                                   \\
        C_p M_\mathrm{W}  = & \left\{
        \begin{alignedat}{3}
            & \num{21.868372} + \num{8.068661e-3} T - \num{3.756196e-6} T^2 +     \\
            & + \num{1.075862e-9} T^3 + \num{1.406637e4} T^{-2}, \quad \SI{300}{\kelvin} \leq T \leq \SI{3080}{\kelvin}, \\
            & \num{2.022} + \num{1.315e-3} T, \quad \SI{3080}{\kelvin} \textless T \leq \SI{3695}{\kelvin},
        \end{alignedat}
        \right.                                                                                              \\
        \kappa =            & \num{149.441} - \num{45.466e-3} T + \num{13.193e-6} T^2 - \num{1.484e-9} T^3 + \\
                            & + \num{3.866e6} T^{-2}, \nonumber
    \end{align}
\end{subequations}
где \( T \) "--- температура, \si{\kelvin}; \( M_\mathrm{W}=\SI{183.84e-3}{\kilogram\per\mole} \) "--- молярная масса вольфрама.

\begin{comment}
Для бериллия использовались функциональные зависимости, рекомендованные Толиасом~\cite{Tolias2022} на основе критического анализа литературных данных. Изменением теплофизических свойств при переходе из альфа- в бета-фазу пренебрегалось:
\begin{subequations}
    \label{eq:app/Be_props}
    \begin{align}
        \rho              & =  \num{1850},                                                                         \\
        C_p M_\mathrm{Be} & =  \num{21.205} + \num{5.694e-3} T + \num{0.962e-6} T^2 - \num{0.5874e6} T^{-2},       \\
        \kappa            & =  \num{148.8912} - \num{76.3780e-3} T + \num{12.0174e-6} T^2 + \num{6.5407e6} T^{-2},
    \end{align}
\end{subequations}
где \( M_\mathrm{Be}=\SI{9.012e-3}{\kilogram\per\mole} \) "--- молярная масса бериллия.
\end{comment}

Функциональные зависимости для меди и сплава бронзы (CuCrZr) были выбраны на основе работы Ван ден Керкхофа и др.~\cite{VandenKerkhof2021}, которые получили аналитические зависимости на основе данных из внутреннего отчета Международной организации ИТЭР~\cite{komarov2013thermal}. Зависимости для меди определены следующим образом:
\begin{subequations}
    \label{eq:app/Cu_props}
    \begin{align}
        \rho   & =  \num{8960},                                                             \\
        C_p    & =  \num{3.71e2} + \num{4.59e-2} T + \num{3.85e-5} T^2 ,                    \\
        \kappa & =  \num{4.18e2} - \num{4.75e-2} T - \num{4.81e-5} T^2 + \num{3.09e-8} T^3.
    \end{align}
\end{subequations}

Соответствующие зависимости для сплава бронзы:
\begin{subequations}
    \label{eq:app/CuCrZr_props}
    \begin{align}
        \rho   & =  \num{8940},                                                             \\
        C_p    & =  \num{3.63e2} + \num{8.96e-2} T + \num{7.58e-6} T^2 ,                    \\
        \kappa & = \num{1.84e2} + \num{7.28e-1} T - \num{1.08e-3} T^2 +  \num{5.26e-7} T^3.
    \end{align}
\end{subequations}

\chapter{Параметры моделирования для проведения валидационных тестов}\label{app:B}

\begin{table}[ht]
    \centering
    \begin{threeparttable}
        \caption{Параметры моделирования для первого валидационного теста}
        \label{tab:case1_inputs}
        \renewcommand{\arraystretch}{1.2}%% Увеличение расстояния между рядами, для улучшения восприятия.
        \begin{tabularx}{\textwidth}{@{}>{\raggedright}Xcc}
            \toprule
            Наименование                                                                                                                    & Обозначение           & Значение       \\
            \hline
            \hline
            Толщина области, \si{\meter}                                                                                                    & $L$                   & \num{1.0e-3}   \\
            Концентрация атомов решетки титана,~\si{\per\meter\cubed}                                                                       & $n_\mathrm{Ti}$       & \num{5.66e28}  \\
            Концентрация межузельных положений,~\si{\per\meter\cubed}                                                                       & $n_\mathrm{IS}$       & \num{1.69e29}  \\
            Концентрация адсорбционных положений,~\si{\per\meter\squared}                                                                   & $n_\mathrm{s}$        & \num{3.90e19}  \\
            Расстояние между межузельными положениями,~\si{\meter}                                                                          & $\lambda_\mathrm{IS}$ & \num{2.29e-10} \\
            Предэкспоненциальный множитель коэффициента диффузии,~\si{\meter\squared\per\second}                                            & $D_0$                 & \num{9.0e-7}   \\
            Энергия активации диффузии,~\si{\electronvolt}                                                                                  & $E_\mathrm{D}$        & \num{0.538}    \\
            Энергия активации десорбции,~\si{\electronvolt}                                                                                 & $E_\mathrm{des}$      & \num{0.562}    \\
            Энергия активации перехода из растворенного состояния в адсорбированное,~\si{\electronvolt}                                     & $E_\mathrm{bs}$       & \num{1.052}    \\
            Энергия активации перехода из адсорбированного состояния в растворенное,~\si{\electronvolt}                                     & $E_\mathrm{sb}$       & \num{1.009}    \\
            Энергия активации диссоциативной адсорбции,~\si{\electronvolt}                                                                  & $E_\mathrm{diss}$     & \num{2.06e-2}  \\
            Предэкспоненциальный множитель коэффициента прилипания                                                                          & $\Phi_0$              & \num{1.43e-2}  \\
            Предэкспоненциальный множитель константы скорости десорбции,~\si{\meter\squared\per\second}                                     & $\nu_\mathrm{des,0}$  & \num{3.41e-11} \\
            Предэкспоненциальный множитель константы скорости перехода из растворенного состояния в адсорбированное,~\si{\meter\per\second} & $\nu_\mathrm{bs,0}$   & \num{2.3}      \\
            Предэкспоненциальный множитель константы скорости перехода из адсорбированного состояния в растворенное,~\si{\per\second}       & $\nu_\mathrm{sb,0}$   & \num{5.14e9}   \\
            Начальное давление,~\si{\pascal}                                                                                                & $P_\mathrm{H_2,0}$    & \num{1.3e4}    \\
            Объем камеры,~\si{\meter\cubed}                                                                                                 & $V_\mathrm{ch}$       & \num{2.95e-3}  \\
            \bottomrule
        \end{tabularx}
    \end{threeparttable}
\end{table}

\begin{table}[t!]
    \centering
    \begin{threeparttable}
        \caption{Параметры моделирования для второго валидационного теста}
        \label{tab:case2_inputs}
        \renewcommand{\arraystretch}{1.2}%% Увеличение расстояния между рядами, для улучшения восприятия.
        \begin{tabularx}{\textwidth}{@{}>{\raggedright}Xcc}
            \toprule
            Наименование                                                                                                & Обозначение           & Значение                     \\
            \hline
            \hline
            Толщина области, \si{\meter}                                                                                & $L$                   & \num{2,0e-3}                 \\
            Концентрация атомов решетки вольфрама,~\si{\per\meter\cubed}                                                & $n_\mathrm{W}$        & \num{6.3382e28}              \\
            Концентрация межузельных положений,~\si{\per\meter\cubed}                                                   & $n_\mathrm{IS}$       & \num{3.8029e29}              \\
            Концентрация адсорбционных положений на чистой поверхности ($\theta_\mathrm{O}=0$),~\si{\per\meter\squared} & $n_\mathrm{s}$        & \num{1.416e19}               \\
            Расстояние между межузельными положениями,~\si{\meter}                                                      & $\lambda_\mathrm{IS}$ & \num{1.117e-10}              \\
            Частота тепловых колебаний атомов,~\si{\per\second}                                                         & $\nu_0$               & \num{e13}                    \\
            Предэкспоненциальный множитель коэффициента диффузии,~\si{\meter\squared\per\second}                        & $D_0$                 & \num{1.365e-7}               \\
            Энергия активации диффузии,~\si{\electronvolt}                                                              & $E_\mathrm{D}$        & \num{0.2}                    \\
            Энергия активации десорбции,~\si{\electronvolt}                                                             & $E_\mathrm{des}$      & ур.~\cref{eq:ch2/case2_Edes} \\
            Энергия активации перехода из растворенного состояния в адсорбированное,~\si{\electronvolt}                 & $E_\mathrm{bs}$       & \num{0.2}                    \\
            Энергия активации перехода из адсорбированного состояния в растворенное,~\si{\electronvolt}                 & $E_\mathrm{sb}$       & ур.~\cref{eq:ch2/case2_Esb}  \\
            Энергия активации диссоциативной адсорбции,~\si{\electronvolt}                                              & $E_\mathrm{diss} $    & 0                            \\
            Теплота растворения,~\si{\electronvolt}                                                                     & $Q_\mathrm{s}$        & 1                            \\
            Предэкспоненциальный множитель коэффициента прилипания                                                      & $\Phi_0$              & 1                            \\
            Начальное давление,~\si{\pascal}                                                                            & $P_\mathrm{D_2}$      & \num{2e-5}                   \\
            \bottomrule
        \end{tabularx}
    \end{threeparttable}
\end{table}

\begin{table}[t!]
    \centering
    \begin{threeparttable}
        \caption{Параметры, определяющие функциональную зависимость энергии активации десорбции от степени покрытия поверхности вольфрама дейтерием}
        \label{tab:case2_Edes_params}
        \renewcommand{\arraystretch}{1.2}%% Увеличение расстояния между рядами, для улучшения восприятия.
        \begin{tabularx}{\textwidth}{>{\centering\arraybackslash}X>{\centering\arraybackslash}X>{\centering\arraybackslash}X>{\centering\arraybackslash}X>{\centering\arraybackslash}X>{\centering\arraybackslash}X>{\centering\arraybackslash}X}
            \toprule
            {\(\theta_{\mathrm{O}}\)} & {$E_0$,~\si{\electronvolt}} & {$\Delta E$,~\si{\electronvolt}} & {$\theta_\mathrm{D,0}$} & {$\delta\theta_\mathrm{D}$} & {a}   & {b}   \\
            \hline
            \hline
            0.00                      & 1.142                       & 0.346                            & 0.253                   & 0.180                       & 0.303 & 8.902 \\
            0.50                      & 1.111                       & 0.289                            & 0.113                   & 0.082                       & 0.460 & 7.240 \\
            0.75                      & 1.066                       & 0.234                            & 0.161                   & 0.057                       & 0.437 & 4.144 \\
            \bottomrule
        \end{tabularx}
    \end{threeparttable}
\end{table}

\begin{table}[t!]
    \centering
    \begin{threeparttable}
        \caption{Параметры моделирования для третьего валидационного теста}
        \label{tab:case3_inputs}
        \renewcommand{\arraystretch}{1.2}%% Увеличение расстояния между рядами, для улучшения восприятия.
        \begin{tabularx}{\textwidth}{@{}>{\raggedright}Xcc}
            \toprule
            Наименование                                                                                & Обозначение           & Значение        \\
            \hline
            \hline
            Толщина области, \si{\meter}                                                                & $L$                   & \num{0.800e-3}  \\
            Концентрация атомов решетки вольфрама,~\si{\per\meter\cubed}                                & $n_\mathrm{W}$        & \num{6.300e28}  \\
            Концентрация межузельных положений,~\si{\per\meter\cubed}                                   & $n_\mathrm{IS}$       & \num{3.780e29}  \\
            Концентрация адсорбционных положений,~\si{\per\meter\squared}                               & $n_\mathrm{s}$        & \num{1.090e20}  \\
            Расстояние между межузельными положениями,~\si{\meter}                                      & $\lambda_\mathrm{IS}$ & \num{1.100e-10} \\
            Частота тепловых колебаний атомов,~\si{\per\second}                                         & $\nu_0$               & \num{1.000e13}  \\
            Предэкспоненциальный множитель коэффициента диффузии,~\si{\meter\squared\per\second}        & $D_0$                 & \num{1.365e-7}  \\
            Энергия активации диффузии,~\si{\electronvolt}                                              & $E_\mathrm{D}$        & \num{0.200}     \\
            Энергия активации десорбции,~\si{\electronvolt}                                             & $E_\mathrm{des}$      & \num{1.740}     \\
            Энергия активации перехода из растворенного состояния в адсорбированное,~\si{\electronvolt} & $E_\mathrm{bs}$       & \num{0.200}     \\
            Энергия активации перехода из адсорбированного состояния в растворенное,~\si{\electronvolt} & $E_\mathrm{sb}$       & \num{1.545}     \\
            Энергия активации диссоциативной адсорбции,~\si{\electronvolt}                              & $E_\mathrm{diss}$     & \num{0.63}      \\
            Коэффициент прилипания                                                                      & $\Phi$                & \num{0.190}     \\
            Плотность потока атомов,~\si{\per\meter\squared\per\second}                                 & $\Gamma_\mathrm{D}$   & \num{5.800e18}  \\
            Сечение рекомбинации Или-Ридила,~\si{\meter\squared}                                        & $\sigma_\mathrm{ER}$  & \num{1.700e-21} \\
            \bottomrule
        \end{tabularx}
    \end{threeparttable}
\end{table}

\begin{table}[t!]
    \centering
    \begin{threeparttable}
        \caption{Параметры центров захвата для третьего валидационного теста}
        \label{tab:case3_traps_params}
        \renewcommand{\arraystretch}{1.2}%% Увеличение расстояния между рядами, для улучшения восприятия.
        \begin{tabularx}{\textwidth}{>{\centering\arraybackslash}X>{\centering\arraybackslash}X>{\centering\arraybackslash}X>{\centering\arraybackslash}X>{\centering\arraybackslash}X>{\centering\arraybackslash}X}
            \toprule
            Тип дефекта & $n_\mathrm{t}$,~ат.\% & $E_\mathrm{dt}$,~\si{\electronvolt} & $\nu_\mathrm{dt,0}$,~\si{\per\second} & $E_\mathrm{t}$,~\si{\electronvolt} & $\nu_\mathrm{t,0}$,~\si{\meter\cubed\per\second} \\
            \hline
            \hline
            1           & \num{0.01}            & \num{0.85}                          & \multirow{5}{*}{\num{e13}}            & \multirow{5}{*}{\num{0.2}}         & \multirow{5}{*}{\num{2.98e-17}}                  \\
            2           & \num{0.01}            & \num{1.00}                          &                                       &                                    &                                                  \\
            3           & \num{0.19}            & \num{1.65}                          &                                       &                                    &                                                  \\
            4           & \num{0.16}            & \num{1.85}                          &                                       &                                    &                                                  \\
            5           & \num{0.02}            & \num{2.06}                          &                                       &                                    &                                                  \\
            \bottomrule
        \end{tabularx}
    \end{threeparttable}
\end{table}

\begin{table}[t!]
    \centering
    \begin{threeparttable}
        \caption{Параметры моделирования для четвертого валидационного теста}
        \label{tab:case4_inputs}
        \renewcommand{\arraystretch}{1.2}%% Увеличение расстояния между рядами, для улучшения восприятия.
        \begin{tabularx}{\textwidth}{@{}>{\raggedright}Xcc}
            \toprule
            Наименование                                                                                                                    & Обозначение                           & Значение        \\
            \hline
            \hline
            Толщина области, \si{\meter}                                                                                                    & $L$                                   & \num{0.800e-3}  \\
            Концентрация атомов решетки вольфрама,~\si{\per\meter\cubed}                                                                    & $n_\mathrm{EFe}$                      & \num{8.59e28}   \\
            Концентрация межузельных положений,~\si{\per\meter\cubed}                                                                       & $n_\mathrm{IS}$                       & \num{5.15e29}   \\
            Концентрация адсорбционных положений,~\si{\per\meter\squared}                                                                   & $n_\mathrm{s}$                        & \num{1.947e20}  \\
            Расстояние между межузельными положениями,~\si{\meter}                                                                          & $\lambda_\mathrm{IS}$                 & \num{2.266e-10} \\
            Предэкспоненциальный множитель коэффициента диффузии,~\si{\meter\squared\per\second}                                            & $D_0$                                 & \num{1.5e-7}    \\
            Энергия активации диффузии,~\si{\electronvolt}                                                                                  & $E_\mathrm{D}$                        & \num{0.15}      \\
            Энергия активации десорбции,~\si{\electronvolt}                                                                                 & $E_\mathrm{des}$                      & \num{0.63}      \\
            Энергия активации перехода из растворенного состояния в адсорбированное,~\si{\electronvolt}                                     & $E_\mathrm{bs}$                       & \num{0.200}     \\
            Энергия активации диссоциативной адсорбции,~\si{\electronvolt}                                                                  & $E_\mathrm{diss}$                     & \num{0.4}       \\
            Теплота растворения,~\si{\electronvolt}                                                                                         & $Q_\mathrm{s}$                        & \num{0.238}     \\
            Предэкспоненциальный множитель коэффициента прилипания                                                                          & $\Phi_0$                              & \num{e-4}       \\
            Предэкспоненциальный множитель растворимости,~\si{\per\meter\cubed\per\pascal}$^{-0.5}$                                         & $K_{\mathrm{S},0}$                    & \num{1.319e23}  \\
            Предэкспоненциальный множитель константы скорости перехода из растворенного в адсорбированное состояние,~\si{\meter\per\second} & $\nu_{\mathrm{bs},0}$                 & \num{661.844}   \\
            Предэкспоненциальный множитель константы скорости десорбции,~\si{\meter\squared\per\second}                                     & $\nu_{\mathrm{des},0}^{\mathrm{D_2}}$ & \num{661.844}   \\
            Плотность потока атомов,~\si{\per\meter\squared\per\second}                                                                     & $\Gamma_\mathrm{D}$                   & \num{5.5e18}    \\
            Сечение рекомбинации Или-Ридила,~\si{\meter\squared}                                                                            & $\sigma_\mathrm{ER}$                  & \num{4.109e-15} \\
            Глубина внедрения,~\si{\meter}                                                                                                  & $X$                                   & \num{-e-10}     \\
            Стандартное отклонение профиля внедренных атомов,~\si{\meter}                                                                   & $\sigma$                              & \num{7.5e-10}   \\
            Коэффициент отражения                                                                                                           & $r$                                   & \num{0.612}     \\
            Начальное давление,~\si{\pascal}                                                                                                & $P_\mathrm{D_2}$                      & \num{1}         \\
            \bottomrule
        \end{tabularx}
    \end{threeparttable}
\end{table}

\begin{table}[t!]
    \centering
    \begin{threeparttable}
        \caption{Параметры центров захвата для четвертого валидационного теста}
        \label{tab:case4_traps_params}
        \renewcommand{\arraystretch}{1.2}%% Увеличение расстояния между рядами, для улучшения восприятия.
        \begin{tabularx}{\textwidth}{>{\centering\arraybackslash}X>{\centering\arraybackslash}X>{\centering\arraybackslash}X>{\centering\arraybackslash}X>{\centering\arraybackslash}X>{\centering\arraybackslash}X}
            \toprule
            Тип дефекта
             & $n_\mathrm{t}$,~ат.\%
             & $E_\mathrm{dt}$,~\si{\electronvolt}
             & $\nu_\mathrm{dt,0}$,~\si{\per\second}
             & $E_\mathrm{t}$,~\si{\electronvolt}
             & $\nu_\mathrm{t,0}$,~\si{\meter\cubed\per\second}      \\
            \hline
            \hline
            1
             & \num{1.0e-3}
             & \num{0.90}
             & \multirow{2}{*}{\num{4.00e13}}
             & \multirow{2}{*}{\num{0.15}}
             & \multirow{2}{*}{\num{5.67e-18}}                       \\
            2
             & \numrange{2.5}{5.0e-2}
             & \num{1.08}
             &                                                  &  & \\
            \bottomrule
        \end{tabularx}
    \end{threeparttable}
\end{table}

\chapter{Физические параметры, использованные для моделирования транспорта дейтерия в вольфраме}\label{app:C}

\begin{table}[ht]
    \centering
    \begin{threeparttable}
        \caption{Физические параметры вольфрама}
        \label{tab:W_props}
        \renewcommand{\arraystretch}{1.2}%% Увеличение расстояния между рядами, для улучшения восприятия.
        \begin{tabularx}{\textwidth}{@{}>{\raggedright}Xcc}
            \toprule
            Наименование                                                                                     & Обозначение            & Значение                    \\
            \hline
            \hline
            Концентрация атомов решетки,~\si{\per\meter\cubed}                                               & $n_\mathrm{W}$         & \num{6.310e28}              \\
            Концентрация межузельных положений,~\si{\per\meter\cubed}                                        & $n_\mathrm{IS}$        & \num{3.786e29}              \\
            Концентрация адсорбционных положений,~\si{\per\meter\squared}                                    & $n_\mathrm{s}$         & \num{1.997e19}              \\
            Расстояние между межузельными положениями,~\si{\meter}                                           & $\lambda_\mathrm{IS}$  & \num{1.100e-10}             \\
            Расстояние между межузельным и адсорбционным положениями,~\si{\meter}                            & $\lambda_\mathrm{abs}$ & \num{5.275e-11}             \\
            Предэкспоненциальный множитель коэффициента диффузии,~\si{\meter\squared\per\second}             & $D_0$                  & \num{1.365e-7}              \\
            Частота тепловых колебаний атомов,~\si{\per\second}                                              & $\nu_0$                & \num{1.000e13}              \\
            Предэкспоненциальный множитель коэффициента диффузии,~\si{\meter\squared\per\second}             & $D_0$                  & \num{1.365e-7}              \\
            Предэкспоненциальный множитель константы скорости захвата в дефект,~\si{\meter\cubed\per\second} & $\nu_{\mathrm{t},0}$   & \num{2.979e-17}             \\
            Предэкспоненциальный множитель константы скорости выхода из дефекта,~\si{\per\second}            & $\nu_{\mathrm{dt},0}$  & \num{1.000e13}              \\
            Предэкспоненциальный множитель константы скорости перехода из растворенного
            в адсорбированное состояние,~\si{\meter\per\second}                                              & $\nu_{\mathrm{bs},0}$  & \num{594.931}               \\
            Предэкспоненциальный множитель константы скорости перехода из адсорбированного
            в растворенное состояние,~\si{\per\second}                                                       & $\nu_{\mathrm{sb},0}$  & \num{1.000e13}              \\
            Энергия активации диффузии,~\si{\electronvolt}                                                   & $E_\mathrm{D}$         & \num{0.200}                 \\
            Теплота растворения,~\si{\electronvolt}                                                          & $Q_\mathrm{s}$         & \num{1.040}                 \\
            Энергия активации перехода из растворенного состояния в адсорбированное,~\si{\electronvolt}      & $E_\mathrm{bs}$        & \num{0.200}                 \\
            Энергия активации перехода из адсорбированного состояния в растворенное,~\si{\electronvolt}      & $E_\mathrm{sb}$        & $E_\mathrm{s}-Q_\mathrm{c}$ \\
            \bottomrule
        \end{tabularx}
    \end{threeparttable}
\end{table}

\begin{comment}
\begin{table}[ht]
    \centering
    \begin{threeparttable}
        \caption{Физические параметры бериллия}
        \label{tab:Be_props}
        \renewcommand{\arraystretch}{1.2}%% Увеличение расстояния между рядами, для улучшения восприятия.
        \begin{tabularx}{\textwidth}{@{}>{\raggedright}Xcc}
            \toprule
            Наименование                                                                                & Обозначение            & Значение        \\
            \hline
            \hline
            Концентрация атомов решетки,~\si{\per\meter\cubed}                                          & $n_\mathrm{Be}$        & \num{1.230e28}  \\
            Концентрация межузельных положений,~\si{\per\meter\cubed}                                   & $n_\mathrm{IS}$        & \num{7.380e29}  \\
            Концентрация адсорбционных положений ,~\si{\per\meter\squared}                              & $n_\mathrm{s}$         & \num{2.473e19}  \\
            Расстояние между межузельными положениями,~\si{\meter}                                      & $\lambda_\mathrm{IS}$  & \num{3.154e-10} \\
            Расстояние между межузельным и адсорбционным положениями,~\si{\meter}                       & $\lambda_\mathrm{abs}$ & \num{3.351e-11} \\
            Частота тепловых колебаний атомов,~\si{\per\second}                                         & $\nu_0$                & \num{1.000e13}  \\
            Предэкспоненциальный множитель коэффициента диффузии,~\si{\meter\squared\per\second}        & $D_0$                  & \num{9.899e-7}  \\
            Энергия активации диффузии,~\si{\electronvolt}                                              & $E_\mathrm{D}$         & \num{0.380}     \\
            Энергия активации перехода из растворенного состояния в адсорбированное,~\si{\electronvolt} & $E_\mathrm{bs}$        & \num{0.380}     \\
            Энергия активации перехода из адсорбированного состояния в растворенное,~\si{\electronvolt} & $E_\mathrm{sb}$        & \num{2.460}     \\
            Теплота растворения,~\si{\electronvolt}                                                     & $Q_\mathrm{s}$         & \num{1.580}     \\
            Теплота хемосорбции,~\si{\electronvolt}                                                     & $|Q_\mathrm{c}|$       & \num{0.500}     \\
            Энергия активации растворения,~\si{\electronvolt}                                             & $E_\mathrm{s}$         & \num{1.960}     \\
            \bottomrule
        \end{tabularx}
    \end{threeparttable}
\end{table}
\end{comment}

\chapter{Параметры, определяющие температурную зависимость потока тепла через границу раздела W-Cu}\label{app:D}

\begin{table}[h]
    \renewcommand{\arraystretch}{1.25}
       \centering
    \rotatebox{270}{
    \begin{minipage}{0.9\textwidth}
        \small
        \caption{Подгоночные параметры в уравнениях~\cref{eq:ch3/heat_WCu}} \label{tab:fit_params}
        \begin{threeparttable}
            \begin{tabular}{*{13}c}
                \toprule
                $q_{\mathrm{stat}}$, & Параметр & \multicolumn{11}{l}{Частота ELM-событий, Гц} \\
                \cline{3-13} 	 
                \si{\mega\watt\per\meter\squared} && Стац. &10 & 20 & 30 & 40 & 50 & 60 & 70 & 80 & 90 & 100 \\
                \hline
                \hline
                1 & a, 1 	& 0.151 & 0.135 & 0.134 & 0.132 & 0.130 & 0.129 & 0.128 & 0.127 & 0.126 & 0.125 & 0.124\\
                & b, К$^{-1}$  & 0.063 & 0.012 & 0.009 & 0.008 & 0.007 & 0.006 & 0.006 & 0.005 & 0.005 & 0.005 & 0.005 \\
                & c, 1 		  & 0.145 & 0.122 & 0.126 & 0.125 & 0.124 & 0.123 & 0.122 & 0.121 & 0.120 & 0.120 & 0.119 \\
                & d, К 		  & 0.96  & 4.16  & 6.05 & 7.02  & 7.82  & 8.49  & 9.11  & 9.65  & 10.18  & 10.65 & 11.09 \\
                        
                5 & a, 1  		  & 0.139 & 0.131 & 0.128 & 0.126 & 0.125 & 0.124 & 0.123 & 0.123 & 0.122 & 0.121 & 0.121\\
                & b, К$^{-1}$  & 0.013 & 0.007 & 0.006 & 0.005 & 0.005 & 0.004 & 0.004 & 0.004 & 0.004 & 0.004 & 0.004 \\
                & c, 1 		  & 0.134 & 0.120 & 0.121 & 0.121 & 0.120 & 0.119 & 0.118 & 0.117 & 0.117 & 0.117 & 0.116 \\
                & d, К 		  & 4.29  & 8.20  & 9.00  & 9.83  & 10.57  & 11.22  & 11.81  & 12.36  & 12.88 & 13.34 & 13.78 \\
                            
                10& a, 1 	      & 0.130 & 0.124 & 0.123 & 0.121 & 0.121 & 0.120 & 0.119 & 0.119 & 0.119 & 0.118 & 0.118\\
                & b, К$^{-1}$  & 0.007 & 0.005 & 0.004 & 0.004 & 0.003 & 0.003 & 0.003 & 0.003 & 0.003 & 0.003 & 0.003 \\
                & c, 1 		  & 0.125 & 0.118 & 0.118 & 0.117 & 0.116 & 0.116 & 0.115 & 0.115 & 0.114 & 0.114 & 0.114 \\
                & d, К 		  & 7.81 & 11.79 & 12.31 & 13.09 & 13.82 & 14.48 & 15.08 & 15.64 & 16.17 & 16.66 & 17.14 \\
                \bottomrule
            \end{tabular}
        \end{threeparttable}
    \end{minipage}}
\end{table}


\chapter{Решение стационарного уравнения диффузии с учетом градиента температуры}\label{app:E}

Рассмотрим стационарное уравнение диффузии с известным распределением температуры ($\partial_x T \neq 0$), определенное на отрезке \( x \in [0, L] \). На левой границе зададим однородное граничное условие Дирихле, когда на правой рассмотрим два случая: однородное условие Дирихле и однородное условие Неймана. Сначала обобщим уравнение~\eqref{eq:ch3/eq:solute_analytic} на случай $k$ источников:
\begin{align}
	\label{eq:general_stat_dif}
	\begin{cases}
		\dfrac{\partial}{\partial x}\left( D(T)\left[ \dfrac{\partial c}{\partial x} + H(T)c\right]\right)=-\sum\limits_{i=1}^k S_i(x), \\[5pt]
		\left.c\right\vert_{x=0}=0,                                                                                                     \\[5pt]
		\left[\begin{array}{ll}
			      \left.c\right\vert_{x=L}=0,  \text{если гр. усл. Дирихле}, \\[5pt]
			      \left.D(T)\left( \dfrac{\partial c}{\partial x} + H(T)c\right)\right\vert_{x=L}=0,  \text{если гр. усл. Неймана}.
		      \end{array}
		\right.
	\end{cases}
\end{align}
По сравнению с уравнением.~\eqref{eq:ch3/eq:solute_analytic}, обозначения сокращены для краткости. Уравнение диффузии может быть сведено к:
\begin{equation}
	\frac{\partial}{\partial x}\left(p(x)\frac{\partial}{\partial x}\left[ ce^{\int\limits_0^x H(T)d\chi}\right]\right)=-\sum\limits_{i=1}^k S_i(x),
\end{equation}
где $p(x)=D(T)e^{-\int\limits_0^x H(T)d\chi}$. Решение этого уравнения можно найти, применив теорему суперпозиции:
\begin{equation}
	\label{eq:ST}
	c=e^{-\int\limits_0^x H(T)d\chi}\,\sum\limits_{i=1}^k \left( \int\limits_0^L G(x,y)S_i(y)dy \right),
\end{equation}
где $G(x,y)$ является функцией Грина, удовлетворяющей следующей вспомогательной задаче:
\begin{align}
	\label{eq:Green_problem}
	\begin{cases}
		\dfrac{\partial}{\partial x}\left(p(x)\,\dfrac{\partial G(x,y)}{\partial x}\right)=-\delta(x-y), \\[5pt]
		\left.G(x,y)\right\vert_{x=0}=0,                                                                 \\[5pt]
		\left[\begin{array}{ll}
			      \left.G(x,y)\right\vert_{x=L}=0,  \text{eсли гр. усл. Дирихле}, \\[5pt]
			      \left.p(x)\,\dfrac{\partial G(x,y)}{\partial x}\right\vert_{x=L}=0,  \text{eсли гр. усл. Неймана},
		      \end{array}\right.
	\end{cases}
\end{align}
где \( \delta(x) \) "--- дельта-функция Дирака. В особой точке выполняются свойства функции Грина краевой задачи:
\begin{gather}
	\label{eq:Green_problem_connect}
	 \left.G(x,y)\right\vert_{x=y^+}=\left.G(x,y)\right\vert_{x=y^-},                                                                                   \\
	 \left.p(x)\,\dfrac{\partial G(x,y)}{\partial x}\right\vert_{x=y^+}-\left.p(x)\,\dfrac{\partial G(x,y)}{\partial x}\right\vert_{x=y^-}=-1.\nonumber
\end{gather}

Для обоих типов условий на правой границе функция Грина может быть выражена следующим образом:
\begin{subequations}
	\label{eq:Green_solution}
	\begin{align}
		 & \text{Если гр. усл. Дирихле}:
		\begin{cases}
			G_1(x,y)=\int\limits_0^x\dfrac{d\chi}{p(\chi)}\,\dfrac{\int\limits_y^L\dfrac{d\chi}{p(\chi)}}{\int\limits_0^L\dfrac{d\chi}{p(\chi)}}, & x\in[0,y], \\[25pt]
			G_2(x,y)=\int\limits_0^y\dfrac{d\chi}{p(\chi)}\,\dfrac{\int\limits_x^L\dfrac{d\chi}{p(\chi)}}{\int\limits_0^L\dfrac{d\chi}{p(\chi)}}, & x\in[y,L]. \\
		\end{cases}\label{eq:Green_solution_D} \\[10pt]
		 & \text{Если гр. усл. Неймана}:
		\begin{cases}
			G_1(x,y)=\int\limits_0^x\dfrac{d\chi}{p(\chi)}, & x\in[0,y], \\[10pt]
			G_2(x,y)=\int\limits_0^y\dfrac{d\chi}{p(\chi)}, & x\in[y,L].
		\end{cases}\label{eq:Green_solution_N}
	\end{align}
\end{subequations}
Если градиент температуры мал ($\partial_x T \rightarrow 0$, $T \rightarrow$ const.), функция Грина сводится к:
\begin{subequations}
	\begin{align}
		 & \text{Если гр. усл. Дирихле}:
		\begin{cases}
			G_1(x,y)=\dfrac{x}{D(T)}\,\dfrac{L-y}{L}, & x\in[0,y], \\[25pt]
			G_2(x,y)=\dfrac{y}{D(T)}\,\dfrac{L-x}{L}, & x\in[y,L], \\
		\end{cases} \\[10pt]
		 & \text{Если гр. усл. Неймана}:
		\begin{cases}
			G_1(x,y)=\dfrac{x}{D(T)}, & x\in[0,y], \\[10pt]
			G_2(x,y)=\dfrac{y}{D(T)}, & x\in[y,L],
		\end{cases}
	\end{align}
\end{subequations}
что в точности соответствует выражениям, полученным ранее~\cite{Denis2022}. Подстановка уравнения~\eqref{eq:Green_solution} в выражение~\eqref{eq:ST} позволяет получить общее решение стационарной задачи диффузии:
\begin{align}
	c= & e^{-\int\limits_0^x H(T)d\chi}\,\sum\limits_{i=1}^k \left( \int\limits_0^x G_2(x,y)S_i(y)dy + \int\limits_x^L G_1(x,y)S_i(y)dy \right).\nonumber
\end{align}

\chapter{Решение уравнения теплопроводности на полубесконечной прямой}\label{app:F}

Уравнение теплопроводности, определяемое на полубесконечной прямой, описывается следующим образом:
\begin{align}
	\label{eq:heat_transfer_semiinfinite}
	\begin{cases}
		\dfrac{\partial T}{\partial t} & = a^2 \dfrac{\partial^2 T}{\partial x^2}, \\[5pt]
		\left.\dfrac{\partial T}{\partial x}\right\vert_{x=0}& = -\dfrac{q(t)}{\kappa},                \\[5pt]
        \left. T \right\vert_{t=0} & = T_0,
	\end{cases}
\end{align}
где \( a = \sqrt{\kappa/C_p\rho} \) "--- температуропроводность. Дифференциальное уравнение в частных производных определено на области \( x \geq 0 \). Тепловые свойства материала предполагаются не зависящими от температуры. Известно, что функция Грина для системы~\cref{eq:heat_transfer_semiinfinite} и изменение температуры за счет плотности потока на границе равны~\cite{Bechtel1975}:
\begin{subequations}
    \begin{gather}
        G(x,t|x',t') = \frac{1}{2a\sqrt{\pi(t-t')}} \left[ \exp \left( -\frac{(x-x')^2}{4a^2(t-t')} \right) + \exp \left( -\frac{(x+x')^2}{4a^2(t-t')} \right) \right],\label{eq:heat_eq_Green}\\
        T(x,t)-T_0 \equiv \Delta T(x,t) = \frac{a^2}{\kappa} \int\limits_0^t G(x,t|0,t')q(t')dt'. \label{eq:deltaT}
    \end{gather}
\end{subequations}
Подставляя функцию Грина~\cref{eq:heat_eq_Green} в уравнение~\cref{eq:deltaT} и полагая \(x=0\), приходим к выражению для временной эволюции изменения температуры поверхности:
\begin{equation}
    \Delta T(0,t) = \frac{1}{\sqrt{\pi\kappa C_p \rho}} \int\limits_0^t \frac{q(t')}{\sqrt{t-t'}}dt'.
\end{equation}

Для использованных временных профилей лазерных импульсов интеграл может быть сведен к табулированным трансцендентным функциям. Временной профиль при наносекундном нагреве явно задавался в соответствии с распределением Гаусса:
\begin{equation}
    q_\mathrm{ns}=\frac{E_0}{\sqrt{2\pi}\sigma_\tau} \exp \left( -\frac{(t-t_0)^2}{2\sigma_\tau^2} \right).
\end{equation} 
Повышение температуры поверхности при такой нагрузке~\cite{Bechtel1975}:
\begin{equation}
    \label{eq:dT_ns}
    \Delta T_\mathrm{ns}(0,t) = \frac{E_0}{\sqrt{2\pi\sigma_\tau \kappa C_p \rho}} \exp \left( -\frac{z^2}{4} \right) D_{-1/2}(-z),
\end{equation}
где \( z = (t-t_0)/\sigma_\tau \), \(D_{-1/2}(x) \) "--- функция параболического цилиндра. Дифференцируя выражение~\cref{eq:dT_ns}, можно показать, что максимальная температура поверхности достигается при \( t_\mathrm{ns}^{\max} \approx t_0 + \num{0.765}\sigma_\tau \), а ее значение, выраженное через полуширину профиля нагрузки (\(\tau_\mathrm{ns}\)), составляет:
\begin{equation}
    \Delta T_\mathrm{ns}(0, t_\mathrm{ns}^{\max}) \approx \num{0.783} \, \frac{2E_0}{\sqrt{\pi \tau_\mathrm{ns} \kappa C_p \rho}}.
\end{equation} 

Для трапециевидного временного профиля (уравнения~\cref{eq:ch3/pulse_form}) временная эволюция изменения температуры поверхности определяется следующей совокупностью выражений:
\begin{subequations}
    \label{eq:dT_usms}
    \begin{align}
        \Delta T_\mathrm{i}(0,t) & = \dfrac{2E_0}{\sqrt{\pi\tau_i \kappa C_p \rho}} h(t), \\
        h(t) & = 
        \begin{cases}
            \begin{alignedat}{3}
            h_1(t) = &\sqrt{\dfrac{t}{\tau_i}} - \sqrt{\dfrac{\tau_\mathrm{le}}{\tau_i}} F\left( \sqrt{\dfrac{t}{\tau_\mathrm{le}}} \right), & 0 \leq t \leq t_1, \\
            h_2(t) = &h_1(t) + \sqrt{\dfrac{\tau_\mathrm{le}}{\tau_i}} \exp \left( -\dfrac{t_1}{\tau_\mathrm{le}} \right) F\left( \sqrt{\dfrac{t-t_1}{\tau_\mathrm{le}}} \right) + &\\
            &+ \sqrt{\dfrac{t-t_1}{\tau_i}} \left[ w_1(t_1) \left( 1+\dfrac{2\Delta}{3}\, \dfrac{t-t_1}{t_2-t_1} \right) -1 \right], & t_1 < t \leq t_2, \\
            h_3(t) =& h_2(t) + w_2(t_2) \sqrt{\dfrac{\tau_\mathrm{te}}{\tau_i}} F\left( \sqrt{\dfrac{t-t_2}{\tau_\mathrm{te}}} \right) - & \\
            &- w_1(t_1) \sqrt{\dfrac{t-t_2}{\tau_i}} \left[ \dfrac{\Delta}{3} \left( 1 + 2 \dfrac{t-t_1}{t_2-t_1} \right) + 1 \right], & t>t_2, 
        \end{alignedat}
        \end{cases}
    \end{align}
\end{subequations}
где \( F(x) \) "--- интеграл Доусона; индекс \(i\) соответствует случаю микросекундного или миллисекундного нагрева. Для параметров импульсов, использованных в настоящей работе, максимальная температура поверхности достигается при \( t_i^{\max} = t_2 \) и ее значение определяется функцией \(h_2(t_2) \):
\begin{equation}
    \Delta T_i^{\max}(0,t_2) \approx \num{0.863} \, \frac{2E_0}{\sqrt{\pi \tau_i \kappa C_p \rho}}.
\end{equation}