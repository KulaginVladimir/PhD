\chapter{Параметры моделирования для проведения валидационных тестов}\label{app:A}

\begin{table}[h]
    \centering
    \begin{threeparttable}
        \caption{Параметры моделирования для первого валидационного теста}
        \label{tab:case1_inputs}
        \renewcommand{\arraystretch}{1.2}%% Увеличение расстояния между рядами, для улучшения восприятия.
        \begin{tabularx}{\textwidth}{@{}>{\raggedright}Xcc}
            \toprule
            Наименование                                                                                                                    & Обозначение           & Значение       \\
            \hline
            \hline
            Толщина области, \si{\meter}                                                                                                    & $L$                   & \num{1.0e-3}   \\
            Концентрация атомов решетки титана,~\si{\per\meter\cubed}                                                                       & $n_\mathrm{Ti}$       & \num{5.66e28}  \\
            Концентрация межузельных положений,~\si{\per\meter\cubed}                                                                       & $n_\mathrm{IS}$       & \num{1.69e29}  \\
            Концентрация адсорбционных положений,~\si{\per\meter\squared}                                                                   & $n_\mathrm{surf}$     & \num{3.90e19}  \\
            Расстояние между межузельными положениями,~\si{\meter}                                                                          & $\lambda_\mathrm{IS}$ & \num{2.29e-10} \\
            Предэкспоненциальный множитель коэффициента диффузии,~\si{\meter\squared\per\second}                                            & $D_0$                 & \num{9.0e-7}   \\
            Энергия активации диффузии,~\si{\electronvolt}                                                                                  & $E_\mathrm{D}$        & \num{0.538}    \\
            Энергия активации десорбции,~\si{\electronvolt}                                                                                 & $E_\mathrm{des}$      & \num{0.562}    \\
            Энергия активации перехода из растворенного состояния в адсорбированное,~\si{\electronvolt}                                     & $E_\mathrm{bs}$       & \num{1.052}    \\
            Энергия активации перехода из адсорбированного состояния в растворенное,~\si{\electronvolt}                                     & $E_\mathrm{sb}$       & \num{1.009}    \\
            Энергия активации диссоциативной адсорбции,~\si{\electronvolt}                                                                  & $E_\mathrm{diss}$     & \num{2.06e-2}  \\
            Предэкспоненциальный множитель коэффициента прилипания                                                                          & $\Phi_0$              & \num{1.43e-2}  \\
            Предэкспоненциальный множитель константы скорости десорбции,~\si{\meter\squared\per\second}                                     & $\nu_\mathrm{des,0}$  & \num{3.41e-11} \\
            Предэкспоненциальный множитель константы скорости перехода из растворенного состояния в адсорбированное,~\si{\meter\per\second} & $\nu_\mathrm{bs,0}$   & \num{2.3}      \\
            Предэкспоненциальный множитель константы скорости перехода из адсорбированного состояния в растворенное,~\si{\per\second}       & $\nu_\mathrm{sb,0}$   & \num{5.14e9}   \\
            Начальное давление,~\si{\pascal}                                                                                                & $P_\mathrm{H_2,0}$    & \num{1.3e4}    \\
            Объем камеры,~\si{\meter\cubed}                                                                                                 & $V_\mathrm{ch}$       & \num{2.95e-3}  \\
            \bottomrule
        \end{tabularx}
    \end{threeparttable}
\end{table}

\begin{table}[h]
    \centering
    \begin{threeparttable}
        \caption{Параметры моделирования для второго валидационного теста}
        \label{tab:case2_inputs}
        \renewcommand{\arraystretch}{1.2}%% Увеличение расстояния между рядами, для улучшения восприятия.
        \begin{tabularx}{\textwidth}{@{}>{\raggedright}Xcc}
            \toprule
            Наименование                                                                                & Обозначение                            & Значение                     \\
            \hline
            \hline
            Толщина области, \si{\meter}                                                                & $L$                                    & \num{2,0e-3}                 \\
            Концентрация атомов решетки вольфрама,~\si{\per\meter\cubed}                                & $n_\mathrm{W}$                         & \num{6.3382e28}              \\
            Концентрация межузельных положений,~\si{\per\meter\cubed}                                   & $n_\mathrm{IS}$                        & \num{3.8029e29}              \\
            Концентрация адсорбционных положений на чистой поверхности ($\theta_\mathrm{O}=0$),~\si{\per\meter\squared}         & $n_\mathrm{surf}$ & \num{1.416e19}               \\
            Расстояние между межузельными положениями,~\si{\meter}                                      & $\lambda_\mathrm{IS}$                  & \num{1.117e-10}              \\
            Частота тепловых колебаний атомов,~\si{\per\second}                                         & $\nu_0$                                & \num{e13}                    \\
            Предэкспоненциальный множитель коэффициента диффузии,~\si{\meter\squared\per\second}        & $D_0$                                  & \num{1.365e-7}               \\
            Энергия активации диффузии,~\si{\electronvolt}                                              & $E_\mathrm{D}$                         & \num{0.2}                    \\
            Энергия активации десорбции,~\si{\electronvolt}                                             & $E_\mathrm{des}$                       & ур.~\cref{eq:ch2/case2_Edes} \\
            Энергия активации перехода из растворенного состояния в адсорбированное,~\si{\electronvolt} & $E_\mathrm{bs}$                        & \num{0.2}                    \\
            Энергия активации перехода из адсорбированного состояния в растворенное,~\si{\electronvolt} & $E_\mathrm{sb}$                        & ур.~\cref{eq:ch2/case2_Esb}  \\
            Энергия активации диссоциативной адсорбции,~\si{\electronvolt}                              & $E_\mathrm{diss} $                     & 0                            \\
            Теплота растворения,~\si{\electronvolt}                                                     & $Q_\mathrm{S}$                         & 1                            \\
            Предэкспоненциальный множитель коэффициента прилипания                                      & $\Phi_0$                               & 1                            \\
            Начальное давление,~\si{\pascal}                                                            & $P_\mathrm{D_2}$                       & \num{2e-5}                   \\
            \bottomrule
        \end{tabularx}
    \end{threeparttable}
\end{table}

\begin{table}[h]
    \centering
    \begin{threeparttable}
        \caption{Параметры, определяющие функциональную зависимость энергии активации десорбции от степени покрытия поверхности вольфрама кислородом}
        \label{tab:case2_Edes_params}
        \renewcommand{\arraystretch}{1.2}%% Увеличение расстояния между рядами, для улучшения восприятия.
        \begin{tabularx}{\textwidth}{>{\centering\arraybackslash}X>{\centering\arraybackslash}X>{\centering\arraybackslash}X>{\centering\arraybackslash}X>{\centering\arraybackslash}X>{\centering\arraybackslash}X>{\centering\arraybackslash}X}
            \toprule
            {\(\theta_{\mathrm{O}}\)} & {$E_0$,~\si{\electronvolt}} & {$\Delta E$,~\si{\electronvolt}} & {$\theta_\mathrm{D,0}$} & {$\delta\theta_\mathrm{D}$} & {a}   & {b}   \\
            \hline
            \hline
            0.00                      & 1.142                       & 0.346                            & 0.253                   & 0.180                       & 0.303 & 8.902 \\
            0.50                      & 1.111                       & 0.289                            & 0.113                   & 0.082                       & 0.460 & 7.240 \\
            0.75                      & 1.066                       & 0.234                            & 0.161                   & 0.057                       & 0.437 & 4.144 \\
            \bottomrule
        \end{tabularx}
    \end{threeparttable}
\end{table}

\begin{table}[h]
    \centering
    \begin{threeparttable}
        \caption{Параметры моделирования для третьего валидационного теста}
        \label{tab:case3_inputs}
        \renewcommand{\arraystretch}{1.2}%% Увеличение расстояния между рядами, для улучшения восприятия.
        \begin{tabularx}{\textwidth}{@{}>{\raggedright}Xcc}
            \toprule
            Наименование                                                                                & Обозначение           & Значение        \\
            \hline
            \hline
            Толщина области, \si{\meter}                                                                & $L$                   & \num{0.800e-3}  \\
            Концентрация атомов решетки вольфрама,~\si{\per\meter\cubed}                                & $n_\mathrm{W}$        & \num{6.300e28}  \\
            Концентрация межузельных положений,~\si{\per\meter\cubed}                                   & $n_\mathrm{IS}$       & \num{3.780e29}  \\
            Концентрация адсорбционных положений,~\si{\per\meter\squared}                               & $n_\mathrm{surf}$     & \num{1.090e20}  \\
            Расстояние между межузельными положениями,~\si{\meter}                                      & $\lambda_\mathrm{IS}$ & \num{1.100e-10} \\
            Частота тепловых колебаний атомов,~\si{\per\second}                                         & $\nu_0$               & \num{1.000e13}  \\
            Предэкспоненциальный множитель коэффициента диффузии,~\si{\meter\squared\per\second}        & $D_0$                 & \num{1.365e-7}  \\
            Энергия активации диффузии,~\si{\electronvolt}                                              & $E_\mathrm{D}$        & \num{0.200}     \\
            Энергия активации десорбции,~\si{\electronvolt}                                             & $E_\mathrm{des}$      & \num{1.740}     \\
            Энергия активации перехода из растворенного состояния в адсорбированное,~\si{\electronvolt} & $E_\mathrm{bs}$       & \num{0.200}     \\
            Энергия активации перехода из адсорбированного состояния в растворенное,~\si{\electronvolt} & $E_\mathrm{sb}$       & \num{1.545}     \\
            Энергия активации диссоциативной адсорбции,~\si{\electronvolt}                              & $E_\mathrm{diss}$     & \num{0.63}      \\
            Коэффициент прилипания                                                                      & $\Phi$                & \num{0.190}     \\
            Плотность потока атомов,~\si{\per\meter\squared\per\second}                                 & $\Gamma_\mathrm{D}$   & \num{5.800e18}  \\
            Сечение рекомбинации Или-Ридила,~\si{\meter\squared}                                        & $\sigma_\mathrm{ER}$  & \num{1.700e-21} \\
            \bottomrule
        \end{tabularx}
    \end{threeparttable}
\end{table}

\begin{table}
    \centering
    \begin{threeparttable}
        \caption{Параметры центров захвата для третьего валидационного теста}
        \label{tab:case3_traps_params}
        \renewcommand{\arraystretch}{1.2}%% Увеличение расстояния между рядами, для улучшения восприятия.
        \begin{tabularx}{\textwidth}{>{\centering\arraybackslash}X>{\centering\arraybackslash}X>{\centering\arraybackslash}X>{\centering\arraybackslash}X>{\centering\arraybackslash}X>{\centering\arraybackslash}X}
            \toprule
            Тип дефекта & $n_\mathrm{t}$,~ат.\% & $E_\mathrm{dt}$,~\si{\electronvolt} & $\nu_\mathrm{dt,0}$,~\si{\per\second} & $E_\mathrm{t}$,~\si{\electronvolt} & $\nu_\mathrm{t,0}$,~\si{\meter\cubed\per\second} \\
            \hline
            \hline
            1           & \num{0.01}            & \num{0.85}                          & \multirow{5}{*}{\num{e13}}            & \multirow{5}{*}{\num{0.2}}         & \multirow{5}{*}{\num{2.98e-17}}                  \\
            2           & \num{0.01}            & \num{1.00}                          &                                       &                                    &                                                  \\
            3           & \num{0.19}            & \num{1.65}                          &                                       &                                    &                                                  \\
            4           & \num{0.16}            & \num{1.85}                          &                                       &                                    &                                                  \\
            5           & \num{0.02}            & \num{2.06}                          &                                       &                                    &                                                  \\
            \bottomrule
        \end{tabularx}
    \end{threeparttable}
\end{table}

\begin{table}[h]
    \centering
    \begin{threeparttable}
        \caption{Параметры моделирования для четвертого валидационного теста}
        \label{tab:case4_inputs}
        \renewcommand{\arraystretch}{1.2}%% Увеличение расстояния между рядами, для улучшения восприятия.
        \begin{tabularx}{\textwidth}{@{}>{\raggedright}Xcc}
            \toprule
            Наименование                                                                                                                    & Обозначение                           & Значение        \\
            \hline
            \hline
            Толщина области, \si{\meter}                                                                                                    & $L$                                   & \num{0.800e-3}  \\
            Концентрация атомов решетки вольфрама,~\si{\per\meter\cubed}                                                                    & $n_\mathrm{EFe}$                      & \num{8.59e28}   \\
            Концентрация межузельных положений,~\si{\per\meter\cubed}                                                                       & $n_\mathrm{IS}$                       & \num{5.15e29}   \\
            Концентрация адсорбционных положений,~\si{\per\meter\squared}                                                                   & $n_\mathrm{surf}$                     & \num{1.947e20}  \\
            Расстояние между межузельными положениями,~\si{\meter}                                                                          & $\lambda_\mathrm{IS}$                 & \num{2.266e-10} \\
            Предэкспоненциальный множитель коэффициента диффузии,~\si{\meter\squared\per\second}                                            & $D_0$                                 & \num{1.5e-7}    \\
            Энергия активации диффузии,~\si{\electronvolt}                                                                                  & $E_\mathrm{D}$                        & \num{0.15}      \\
            Энергия активации десорбции,~\si{\electronvolt}                                                                                 & $E_\mathrm{des}$                      & \num{0.63}      \\
            Энергия активации перехода из растворенного состояния в адсорбированное,~\si{\electronvolt}                                     & $E_\mathrm{bs}$                       & \num{0.200}     \\
            Энергия активации диссоциативной адсорбции,~\si{\electronvolt}                                                                  & $E_\mathrm{diss}$                     & \num{0.4}       \\
            Теплота растворения,~\si{\electronvolt}                                                                                         & $Q_\mathrm{S}$                        & \num{0.238}     \\
            Предэкспоненциальный множитель коэффициента прилипания                                                                          & $\Phi_0$                              & \num{e-4}       \\
            Предэкспоненциальный множитель растворимости,~\si{\per\meter\cubed\per\pascal}$^{-0.5}$                                         & $K_{\mathrm{S},0}$                    & \num{1.319e23}  \\
            Предэкспоненциальный множитель константы скорости перехода из растворенного в адсорбированное состояние,~\si{\meter\per\second} & $\nu_{\mathrm{bs},0}$                 & \num{661.844}   \\
            Предэкспоненциальный множитель константы скорости десорбции,~\si{\meter\squared\per\second}                                     & $\nu_{\mathrm{des},0}^{\mathrm{D_2}}$ & \num{661.844}   \\
            Плотность потока атомов,~\si{\per\meter\squared\per\second}                                                                     & $\Gamma_\mathrm{D}$                   & \num{5.5e18}    \\
            Сечение рекомбинации Или-Ридила,~\si{\meter\squared}                                                                            & $\sigma_\mathrm{ER}$                  & \num{4.109e-15} \\
            Глубина внедрения,~\si{\meter}                                                                                                  & $X$                                   & \num{-e-10}     \\
            Стандартное отклонение профиля внедренных атомов,~\si{\meter}                                                                   & $\sigma$                              & \num{7.5e-10}   \\
            Коэффициент отражения                                                                                                           & $r$                                   & \num{0.612}     \\
            Начальное давление,~\si{\pascal}                                                                                                & $P_\mathrm{D_2}$                      & \num{1}         \\
            \bottomrule
        \end{tabularx}
    \end{threeparttable}
\end{table}

\begin{table}[!t]
    \centering
    \begin{threeparttable}
        \caption{Параметры центров захвата для четвертого валидационного теста}
        \label{tab:case4_traps_params}
        \renewcommand{\arraystretch}{1.2}%% Увеличение расстояния между рядами, для улучшения восприятия.
        \begin{tabularx}{\textwidth}{>{\centering\arraybackslash}X>{\centering\arraybackslash}X>{\centering\arraybackslash}X>{\centering\arraybackslash}X>{\centering\arraybackslash}X>{\centering\arraybackslash}X}
            \toprule
            Тип дефекта & $n_\mathrm{t}$,~ат.\%  & $E_\mathrm{dt}$,~\si{\electronvolt} & $\nu_\mathrm{dt,0}$,~\si{\per\second} & $E_\mathrm{t}$,~\si{\electronvolt} & $\nu_\mathrm{t,0}$,~\si{\meter\cubed\per\second} \\
            \hline
            \hline
            1           & \num{1.0e-3}           & \num{0.90}                          & \multirow{2}{*}{\num{4.00e13}}        & \multirow{2}{*}{\num{0.15}}        & \multirow{2}{*}{\num{5.67e-18}}                  \\
            2           & \numrange{2.5}{5.0e-2} & \num{1.08}                          &                                       &                                    &                                                  \\
            \bottomrule
        \end{tabularx}
    \end{threeparttable}
\end{table}