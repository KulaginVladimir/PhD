\chapter{Десорбция дейтерия из вольфрама при импульсном лазерном нагреве}\label{ch:ch4}

\section{Валидация}\label{sec:ch4/sec1}
\subsection{Детали эксперимента}\label{subsec:ch4/sec1/subsec1}
\subsection{Расчетная модель}\label{subsec:ch4/sec1/subsec2}
\subsection{Сравнение результатов моделирования и эксперимента}\label{subsec:ch4/sec1/subsec3}

\section{Анализ состава потока десорбированных частиц}\label{sec:ch4/sec2}
\subsection{Постановка задачи}\label{subsec:ch4/seс2/subsec1}
\subsection{Аналитический анализ}\label{subsec:ch4/seс2/subsec2}
\subsection{Результаты численного моделирования}\label{subsec:ch4/seс2/subsec3}

\section{Анализ влияния параметров материала на выход дейтерия}\label{sec:ch4/seс3}
\subsection{Постановка задачи}\label{subsec:ch4/seс3/subsec1}
\subsection{Влияние теплопроводности материала}\label{subsec:ch4/seс3/subsec2}
\subsection{Влияние параметров дефектов в вольфраме}\label{subsec:ch4/seс3/subsec3}
\subsection{Влияние градиента температур}\label{subsec:ch4/seс3/subsec4}
\subsection{Режимы десорбции во время лазерно-индуцированной десорбции}\label{sec:ch4/seс4/subsec5}

\section{Выводы к Главе~\ref{ch:ch4}}

\clearpage
